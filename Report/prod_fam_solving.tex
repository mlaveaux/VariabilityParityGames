For solving VPGs we distinguish two general approaches, the first approach is to simply project the VPG to the different configurations and solve all the resulting parity games independently. We call this approach \textit{product} based. Alternatively we solve the VPG \textit{family} based where a VPG is solved in its entirety and similarities are used to improve performance. 

In this next sections we explore family based algorithms, analyse their running time complexity and present the results of experiments conducted to test the performance of the different family based algorithms compared to the product based approach.

In general we can take some existing algorithm to solve parity games with running time complexity $O(T)$ and use the algorithm to solve a VPG product based. For a VPG with configurations $\mathfrak{C}$ this gives a running time complexity of $O(|\mathfrak{C}|T)$.