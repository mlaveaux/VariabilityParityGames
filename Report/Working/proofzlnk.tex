
For the following lemma's consider MPG $G = (V,V_0,V_1,E,\Omega)$ defined over $\mathfrak{C}$ and $\mathfrak{V}$. We write $U$ and $X$ to denote subsets of $V$.
\begin{lemma}
	\label{lem_MPG_MTrap_is_trap}
	The set $U \subseteq V$ is an $\alpha$-MTrap in the game $G$ iff for all $c \in con(V)$ the set $cU$ is an $\alpha$-trap in the game $cG$.
	\begin{proof}
		We first note the following equivalences:
		\begin{itemize}
			\item $(c,v) \in X$ iff $v \in cX$ for some $X \subseteq V$ (using \ref{def_cV}).
			\item $((c,v),(c,w)) \in E$ iff $(v,w) \in cE$ (using \ref{def_cE}).
		\end{itemize}
		We can now rewrite the MTrap definition (\ref{def_MTrap}):
		\begin{align*}
		\forall c \in con(U):\forall v &\in cU:\\
		&v \in cV_\alpha \implies \forall(v,w) \in cE : w \in cU\\
		&\wedge\\
		&v \in cV_{\overline{\alpha}} \implies \exists (v,w) \in cE : w \in cU
		\end{align*}
		Using the trap definition (\ref{def_trap}) on a game $cG$ we can rewrite the MTrap definition to:
		\[ \forall c \in con(U): cU\text{ is an $\alpha$-trap in $cG$} \]
		
		To prove the lemma we first consider the set $U \subseteq V$ that is an $\alpha$-MTrap in game $G$ and some $c \in con(V)$. If $c \in con(U)$ then using our rewritten definition we conclude that $cU$ is an $\alpha$-trap in $cG$. If $c \notin con(U)$ then $cU = \emptyset$, the empty set is an $\alpha$-trap by definition. So in either case $c \in con(V)$ is an $\alpha$-trap in $cG$.
		
		Next we consider $U \subseteq V$ where for every $c \in con(V)$ the set $cU$ is an $\alpha$-trap in the game $cG$. Because $U \subseteq V$ we have for every $c \in con(U)$ the set $cU$ is an $\alpha$-trap in the game $cG$. Using our rewritten MTrap definition we can conclude that $U$ is an $\alpha$-MTrap in $G$.
	\end{proof}
\end{lemma}

\begin{lemma}
	\label{lem_MPG_MAttr_is_Attr}
	$X = \alpha\textit{-MAttr}(G,U)$ iff $cX = \alpha\textit{-Attr}(cG, cU)$ for all $c \in con(U)$.
	\begin{proof}
		Consider the MAttr definition (\ref{def_MAttr}). To calculate $U_{i+1}$ all vertices $(c,v) \in V$ are considered, however $(c,v)$ can only be attracted if there is an edge from $(c,v)$ to $(c,v') \in U_i$. Because two vertices that are connected by an edge have the same configuration we only to consider vertices $(c,v) \in V$ for $c \in U_i$. We can also conclude that $con(U) = con(U_i)$ for any $i \geq 0$.
		
		We can now rewrite the MAttr definition to the following:
		
		For all $c \in con(U)$ we have:
		\[ cU_0 = cU \]
		For $i \geq 0$:
		\begin{align*}
		cU_{i+1} = &\{v \in cV_\alpha\ |\ \exists v' \in cV : v' \in cU_i \wedge (v,v') \in cE \}\\
		\cup &\{v \in cV_{\overline{\alpha}}\ |\ \forall v' \in cV :(v,v') \in cE \implies v' \in cU_i \}
		\end{align*}
		\[ c(\alpha\textit{-MAttr}(G,U)) = cU_k \]
		such that for $k$ we have
		\[ cU_k = cU_{k+1} \]
		
		This definition is identical to the $\alpha$-Attr definition filled in for game $cG$ and set $cU$, therefore the lemma holds.
	\end{proof}
\end{lemma}

\begin{lemma}
	\label{lem_MPG_attr_exc_trap}
	The set $U=V\backslash \alpha{-}MAttr(G,X)$ is an $\alpha$-MTrap in $G$ for any $X \subseteq V$.
	\begin{proof}
		Assume $U$ is not an $\alpha$-MTrap. Using the MTrap definition (\ref{def_MTrap}) we can conclude that there exists an $(c,v) \in U$ such that either:
		\begin{enumerate}
			\item $(c,v) \in V_\alpha \wedge \exists((c,v),(c,w)) \in E: (c,w) \notin U$ or
			\item $(c,v) \in V_{\overline{\alpha}} \wedge \forall((c,v),(c,w)) \in E: (c,w) \notin U$
		\end{enumerate}
		Note that $(c,w) \notin U$ is equivalent to $(c,w) \in \alpha\textit{-MAttr}(G,X)$.
		
		If $(c,v) \in V_\alpha$ then there exists an $((c,v),(c,w)) \in E$ such that $(c,w) \in \alpha\textit{-MAttr}(G,X)$. Using the MTrap definition (\ref{def_MTrap}) the vertex $(c,v)$ will also be attracted and therefore $(c,v) \in \alpha\textit{-MAttr}(G,X)$ and $(c,v) \notin U$ which is a contradiction.
		
		If $(c,v) \in V_{\overline{\alpha}}$ then for all $((c,v),(c,w)) \in E$ we have $(c,w) \in \alpha\textit{-MAttr}(G,X)$. Using the MTrap definition (\ref{def_MTrap}) the vertex $(c,v)$ will also be attracted and therefore $(c,v) \in \alpha\textit{-MAttr}(G,X)$ and $(c,v) \notin U$ which is a contradiction.
		
		In all cases we derive at a contradiction, therefore $U$ is an $\alpha$-MTrap.
	\end{proof}
\end{lemma}

\begin{lemma}
	\label{lem_MPG_sub_trap_is_MPG}
	If $V\backslash U$ is an $\alpha$-MTrap then $G\backslash U$ is an MPG.
\end{lemma}

\begin{theorem}
	Given MPG $G = (V,V_0,V_1,E,\Omega)$ it holds that $(c,v) \in W_\alpha$ resulting from $\textsc{RecursiveMPG}(G)$ iff $v \in Q_\alpha$ resulting from $\textsc{RecursivePG}(cG)$.
		\begin{proof}
			We will compare the MPG algorithm with the original PG algorithm and show that they behave similar. We will apply induction on game $G$ with the IH being equal to the theorem.
			\begin{itemize}
				\item \textbf{Base:} If $\min\{\Omega(v)\ |\ (c,v) \in V \} = \max\{\Omega(v)\ |\ (c,v) \in V \}$ then we return $(V,\emptyset)$ or $(\emptyset, V)$ (depending on the parity). For every $c \in \mathfrak{C}$ we have $\min\{\Omega(v)\ |\ v \in cV \} = \max\{\Omega(v)\ |\ v \in cV \}$ in which case $\textsc{RecursivePG}(cG)$ returns $(cV,\emptyset)$ or $(\emptyset,cV)$ according to the original PG algorithm. Therefore IH holds.
				\item \textbf{Base:} If $V = \emptyset$ then we return $(\emptyset, \emptyset)$, clearly $cV = \emptyset$ for every $c \in \mathfrak{C}$ and $\textsc{RecursivePG}(cG)$ returns $(\emptyset, \emptyset)$.
				\item \textbf{Step} Consider configuration $c \in \mathfrak{C}$. We distinguish two cases:
				\begin{itemize}
					\item Assume there is a vertex $v \in cV$ with the highest priority, ie. $\Omega(v) = h$.
					From lemma \ref{lem_MPG_MAttr_is_Attr} we can conclude that $cA = \alpha\textit{-Attr}(cG,cU)$.
					
					From lemma \ref{lem_MPG_attr_exc_trap} we can conclude that $V\backslash A$ is an $\alpha$-MTrap. Using lemma \ref{lem_MPG_sub_trap_is_MPG} we find that $G\backslash^{\!\!M}A$ is an MPG. We can apply the IH on this MPG, this gives winning sets $W_0'$ and $W_1'$ such that $(cW_0',cW_1') = \textsc{RecursivePG}(c(G\backslash^{\!\!M}A)) = \textsc{RecursivePG}(cG\backslash cA))$.
					
					
					\item Assume there is no vertex $v \in cV$ with the highest priority. 
				\end{itemize}
			\end{itemize}
		\end{proof}
\end{theorem}