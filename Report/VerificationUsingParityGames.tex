Verifying LTSs against a modal $\mu$-calculus formula can be done by solving a \textit{parity game}. This is done by translating an LTS in combination with a formula to a parity game, the solution of the parity game provides the information needed to conclude if the model satisfies the formula. This relation is depicted in figure \ref{fig:ltsverificationusingpg}. This technique is well known and well studied, in this section we will first look at parity games, the translation from LTS and formula to a parity game and finally what we can do with this technique to verify FTS.
\begin{figure}[h]
	\centering
	\includegraphics[scale=0.5]{Diagrams/LTSVerificationUsingPG}
	\caption[LTS verification using PG]{LTS verification using PG}
	\label{fig:ltsverificationusingpg}
\end{figure}


\subsection{Parity games}
\begin{definition}
	\label{def_PG}\cite{Bradfield2018}
	A parity game (PG) is a tuple $(V, V_0, V_1, E, \Omega)$, where:
	\begin{itemize}
		\item $V = V_0 \cup V_1$ and $V_0 \cap V_1 = \emptyset$,
		\item $V_0$ is the set of vertices owned by player $0$,
		\item $V_1$ is the set of vertices owned by player $1$, 
		\item $E \subseteq V \times V$ is the edge relation,
		\item $\Omega :  V \rightarrow \mathbb{N}$ is a priority assignment.
	\end{itemize}
\end{definition}
A parity game is played by players 0 and 1. We write $\alpha \in \{0,1\}$ to denote an arbitrary player. We write $\overline{\alpha}$ to denote $\alpha$'s opponent, ie. $\overline{0} = 1$ and $\overline{1} = 0$.

 A play starts with placing a token on vertex $v \in V$. Player $\alpha$ moves the token if the token is on a vertex owned by $\alpha$, ie. $v \in V_\alpha$. The token can be moved to $w \in V$, with $(v,w) \in E$. A series of moves results in a sequence of vertices, called a path. For path $\pi$ we write $\pi_i$ to denote the $i^{\text{th}}$ vertex in path $\pi$. A play ends when the token is on vertex $v \in V_\alpha$ and $\alpha$ can't move the token anywhere, in this case player $\overline{\alpha}$ wins the play. If the play results in an infinite path $\pi$ then we determine the highest priority that occurs infinitely often in this path, formally
\[ \max\{ p \ |\ \forall_j \exists_i j < i \wedge p = \Omega(\pi_i) \}\] 
If the highest priority is odd then player $1$ wins, if it is even player $0$ wins.
\begin{figure}[h]
	\centering
	\includegraphics[scale=0.3]{Examples/SimplePG/PG}
	\caption[Parity game example]{Parity game example}
	\label{fig:simplepgpg}
\end{figure}

Figure \ref{fig:simplepgpg} shows an example of a parity game. We usually depict the vertices owned by player $0$ by diamonds and vertices owned by player $1$ by boxes, the priority is depicted inside the vertices. If the game starts by placing a token on $v_1$ we can consider the following exemplary paths:
\begin{itemize}
	\item $\pi = v_1v_3v_5$ is won by player $1$ since player $0$ can't move at $v_5$.
	\item $\pi = (v_1v_2)^\omega$ is won by player $1$ since the highest priority occurring infinitely often is 3.
	\item $\pi = v_1v_3(v_4)^\omega$ is won by player $0$ since the highest priority occurring infinitely often is $0$.
\end{itemize}


A strategy for player $\alpha$ is a function $\sigma : V^*V_\alpha \rightarrow V$ that maps a path ending in a vertex owned by player $\alpha$ to the next vertex. Parity games are positionally determined \cite{Bradfield2018}, therefore a strategy $\sigma: V_\alpha \rightarrow V$ that maps the current vertex to the next vertex is sufficient. 

A strategy $\sigma$ for player $\alpha$ is winning from vertex $v$ iff any play that results from following $\sigma$ results in a win for player $\alpha$. The graph can be divided in two partitions $W_0 \subseteq V$ and $W_1 \subseteq V$, called winning sets. Iff $v \in W_\alpha$ then player $\alpha$ has a winnings strategy from $v$. Every vertex in the graph is either in $W_0$ or $W_1$ \cite{Bradfield2018}. Furthermore finite parity games are decidable \cite{Bradfield2018}.


\subsection{Creating parity games}
A parity game can be created from a combination of an LTS and a modal $\mu$-calculus formula. To do this we introduce some auxiliary definitions regarding the modal $\mu$-calculus.

First we introduce the notion of unfolding, a fixpoint formula $\mu X . \varphi$ can be unfolded resulting in formula $\varphi$ where every occurrence of $X$ is replaced by $\mu X . \varphi$, denoted by $\varphi [ X:= \mu X . \varphi]$. A fixpoint formula is equivalent to its unfolding \cite{Bradfield2018}, ie. for some LTS $[\![\mu X . \varphi]\!]^\eta = [\![\varphi[X:=\mu X . \varphi]]\!]^\eta$. The same holds for the fixpoint operator $\nu$.

Next we define the Fischer-Ladner closure for a closed $\mu$-calculus formula 
\cite{STREETT1989249,FISCHER1979194}. The Fischer-Ladner closure of $\varphi$ is the set $\textit{FL}(\varphi)$ of closed formula's containing at least $\varphi$. Furthermore for every formula $\psi$ in $\textit{FL}(\varphi)$ it holds that for every direct subformula $\psi'$ of $\psi$ there is a formula in $\textit{FL}(\varphi)$ that is equivalent to $\psi'$.
\begin{definition}
	\label{def_FLClosure}
	The Fischer-Ladner closure of closed $\mu$-calculus formula $\varphi$ is the smallest set $\textit{FL}(\varphi)$ satisfying the following constraints:
	\begin{itemize}
		\item $\varphi \in \textit{FL}(\varphi)$,
		\item if $\varphi_1 \vee \varphi_2 \in \textit{FL}(\varphi)$ then $\varphi_1 ,\varphi_2 \in \textit{FL}(\varphi)$,
		\item if $\varphi_1 \wedge \varphi_2 \in \textit{FL}(\varphi)$ then $\varphi_1 ,\varphi_2 \in \textit{FL}(\varphi)$,
		\item if $\langle a \rangle \varphi' \in \textit{FL}(\varphi)$ then $\varphi' \in \textit{FL}(\varphi)$,
		\item if $[ a ] \varphi' \in \textit{FL}(\varphi)$ then $\varphi' \in \textit{FL}(\varphi)$,
		\item if $\mu X . \varphi' \in \textit{FL}(\varphi)$ then $\varphi'[X:= \mu X . \varphi'] \in \textit{FL}(\varphi)$ and
		\item if $\nu X . \varphi' \in \textit{FL}(\varphi)$ then $\varphi'[X:= \nu X . \varphi'] \in \textit{FL}(\varphi)$.
		
	\end{itemize}
\end{definition}

Finally we define alternating depth.
\begin{definition}\cite{Bradfield2018}
	The dependency order on bound variables of $\varphi$	is the smallest partial order such that $X \leq_\varphi Y$ if $X$ occurs free in $\sigma Y. \psi$ . The alternation depth of a $\mu$-variable X in formula $\varphi $ is the maximal length of a chain $X_1 \leq_\varphi  \dots \leq_\varphi X_n$ where $X = X_1$, variables $X_1, X_3, \dots$ are $\mu$-variables and variables $X_2, X_4, \dots$ are $\nu$-variables. The alternation depth of a $\nu$-variable is defined similarly. The alternation depth of formula $\varphi$, denoted $adepth(\varphi)$, is the maximum of the alternation depths of the variables bound in $\varphi$, or zero if there are no fixpoints.
\end{definition}
Consider the example formula $\varphi = \nu X. \mu Y. ([ins]Y \wedge [std] X)$ which states that for an LTS with $Act = \{ ins, std\}$ the action \textit{std} must occur infinitely often over all runs. Since $X$ occurs free in $\mu Y. ([ins] Y \wedge [std]X)$ we have $adepth(Y) = 1$ and $adepth(X) = 2$. As shown in \cite{Bradfield2018} it holds that formula $\mu X. \psi$ has the same alternation depth as its unfolding $\psi[X:=\mu X. \psi]$. Similarly for the greatest fixpoint.



We can now define the transformation from an LTS and a formula to a parity game.
\begin{definition}
	\label{def_LTS2PG}\cite{Bradfield2018}
	LTS2PG($M, \varphi$) converts LTS $M = (S, Act, trans, s_0)$ and closed formula $\varphi$ to a PG $(V, V_0, V_1, E, \Omega)$.
	
	A vertex in the parity game is represented by a pair $(s, \psi)$ where $s \in S$ and $\psi$ is a modal $\mu$-calculus formula. We will create a vertex for every state with every formula in the Fischer-Ladner closure of $\varphi$. We define the set of vertices:
	\[ V = S \times \textit{FL}(\varphi) \]
	
	We create the parity game with the smallest set $E$ such that:
	\begin{itemize}
		\item $V = V_0 \cup V_1$,
		\item $V_0 \cap V_1 = \emptyset$ and
		\item for every $v = (s, \psi) \in V$ we have:
		\begin{itemize}
			\item If $\psi = \top$ then $v \in V_1$.
			\item If $\psi = \bot$ then $v \in V_0$.
			\item If $\psi = \psi_1 \vee \psi_2$ then:
			\subitem $v \in V_0$,
			\subitem $(v, (s,\psi_1)) \in E$ and
			\subitem $(v, (s,\psi_2)) \in E$.
			\item If $\psi = \psi_1 \wedge \psi_2$ then:
			\subitem $v \in V_1$,
			\subitem $(v, (s,\psi_1)) \in E$ and
			\subitem $(v, (s,\psi_2)) \in E$.
			\item If $\psi = \langle a \rangle \psi'$ then $v \in V_0$ and for every $s \xrightarrow{ a} s'$ we have $(v, (s', \psi')) \in E$.
			\item If $\psi = [ a ] \psi'$ then $v \in V_1$ and for every $s \xrightarrow{ a} s'$ we have  $(v, (s', \psi')) \in E$.
			\item If $\psi = \mu X. \psi'$ then $(v, (s, \psi'[X:=\mu X. \psi'])) \in E$.
			\item If $\psi = \nu X. \psi'$ then $(v, (s, \psi'[X:=\nu X. \psi'])) \in E$.
		\end{itemize}
	\end{itemize}
	Since the Fischer-Ladner formula's are closed we never get the case $\psi = X$.
	
	Finally we have $\Omega(s, \psi) = \begin{cases}
	2 \lfloor adepth(X) / 2 \rfloor & \text{if } \psi = \nu X. \psi'\\
	2 \lfloor adepth(X) / 2 \rfloor + 1 & \text{if } \psi = \mu X. \psi'\\
	0 & \text{otherwise}
	\end{cases}$
\end{definition}
\begin{figure}[h]
	\centering
	\includegraphics[scale=0.3]{Examples/ExamleVerification/LTSprojempty}
	\caption[LTS $M$]{LTS $M$}
	\label{fig:exverltsprojempty}
\end{figure}\begin{figure}[h]
\centering
\includegraphics[scale=0.3]{Examples/ExamleVerification/PG}
\caption[Parity game $LTS2PG(M, \varphi)$]{Parity game $LTS2PG(M, \varphi)$}
\label{fig:exverpg}
\end{figure}
Consider LTS $M$ in figure \ref{fig:exverltsprojempty} and formula $\varphi = \mu X.([a]X \vee \langle b \rangle \top)$ expressing that on any path reached by $a$'s we can eventually do a $b$ action. We will use this as a working example in the next few sections. The resulting parity game is depicted in figure \ref{fig:exverpg}. Solving this parity game results in the following winning sets:%
\begin{align*}
	W_0 = \{& (s_1, \mu X.\phi),\\
	& (s_1, [a](\mu X. \phi) \vee \langle b \rangle \top),\\
	& (s_1, [a](\mu X. \phi)),\\
	& (s_1, \top),\\
	& (s_2, \mu X.\phi),\\
	& (s_2, [a](\mu X. \phi) \vee \langle b \rangle \top),\\
	& (s_2, [a](\mu X. \phi)),\\
	& (s_2, \langle b \rangle \top),\\
	& (s_2, \top)
	\}\\
	W_1 = \{& (s_1, \langle b \rangle \top )\}
\end{align*}
With the strategies $\sigma_0$ for player $0$ and $\sigma_1$ for player $1$ being (vertices with one outgoing edge are omitted):
\begin{align*}
\sigma_0 = \{
&(s_1, [a](\mu X. \phi) \vee \langle b \rangle \top) \mapsto (s_1, [a] (\mu X. \phi)), \\
&(s_2, [a](\mu X. \phi) \vee \langle b \rangle \top) \mapsto (s_2, \langle b \rangle \top) \} \\
\sigma_1 = \{&\} \\
\end{align*}

State $s$ in LTS $M$ only satisfies $\varphi$ iff player $0$ has a winning strategy from vertex $(s, \varphi)$. This is formally stated in the following theorem which is proven in \cite{Bradfield2018}.
\begin{theorem}
	\label{the_LTS_PG_REL}Given LTS $M = (S, Act, trans, s_0)$, modal $\mu$-calculus formula $\varphi$ and state $s \in S$ it holds that $(M, s) \models \varphi$ iff $s \in W_0$ for the game $LTS2PG(M, \varphi)$.
\end{theorem}

\subsection{FTSs and parity games}
Using the theory we have seen thus far we can verify FTSs by verifying every projection of the FTS to a valid product. This relation is depicted in the following diagram where $\Pi$ indicates a projection:
\\\begin{tikzpicture}
\matrix (m) [matrix of math nodes,row sep=4em,column sep=4em,minimum width=2em]
{
	\text{FTS} \\
	\text{LTS} & \text{PG} \\};
\path[-stealth]
(m-1-1) edge [double] node [left] {$\Pi$} (m-2-1)
(m-2-1.east|-m-2-2) edge node [above] {$\varphi$}
(m-2-2)
;
\end{tikzpicture}\\
As mentioned before verifying products dependently is potentially more efficient. In the next two sections we define an extension to parity games, namely \textit{variability parity games} (VPGs) which can be used to verify an FTS. We will translate an FTS and a formula into a VPG which solution will provide the information needed to conclude for which products the FTS satisfies the formula.
