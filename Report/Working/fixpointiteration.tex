\documentclass[]{article}
\usepackage[margin=1in]{geometry}
\usepackage{amsthm}
\usepackage{mathtools}
\usepackage{amsfonts}
\usepackage{tikz}
\usepackage{eurosym}
\usepackage{algorithm}
\usepackage{algorithmicx}
\usepackage{algpseudocode}
\usepackage{multicol}

\DeclareRobustCommand{\officialeuro}{%
	\ifmmode\expandafter\text\fi
	{\fontencoding{U}\fontfamily{eurosym}\selectfont e}}
\usepackage{caption}
\usepackage{subcaption}
\usetikzlibrary{matrix}

\usepackage{stmaryrd}
\newtheorem{definition}{Definition}[section]
\newtheorem{theorem}{Theorem}[section]
\newtheorem{lemma}[theorem]{Lemma}



%opening
\title{Verifying Featured Transition Systems using Variability Parity Games}
\author{Sjef van Loo}

\begin{document}
	Given parity game $G = (V,V_0,V_1,E,\Omega)$ and pre-solved vertices $P_0 \subseteq V$ and $P_1 \subseteq V$ such that $P_0 \subseteq W_0 \subseteq V\backslash P_1$. The formula
	\[ S(G) = \nu Z_{d-1}.\mu Z_{d-2}\dots \nu Z_0.F_0(G,Z_{d-1},\dots,Z_0) \]
	\[ F_0(G,Z_{d-1},\dots,Z_0) = \{v \in V_0\ |\ \exists_{w\in V}\ (v,w) \in E \wedge w\in Z_{\Omega(w)}\} \cup \{ v\in V_1\ |\ \forall_{w\in V}\ (v,w) \in E \implies w\in Z_{\Omega(w)}\} \]
	solves $W_0$ for $G$.
	
	In this section we prove that formula 
	\[ S^P(G) = \nu Z_{d-1}.\mu Z_{d-2}\dots \nu Z_0.(F_0(G,Z_{d-1},\dots,Z_0) \cap (V\backslash P_1) \cup P_0) \]
	also solves $W_0$ for $G$. Note that the formula $F_0(G,Z_{d-1},\dots,Z_0) \cap (V\backslash P_1) \cup P_0$ is still monotonic as shown in Lemma \ref{lem_monotonic_union}.
	\begin{lemma}
		\label{lem_monotonic_union}
		Given lattice $\langle 2^D, \subseteq \rangle $, monotonic function $f :  2^D \rightarrow 2^D$ and $A \subseteq D$. The functions $f^\cup(x) = f(x) \cup A$ and $f^\cap(x) = f(x) \cap A$ are also monotonic.
		\begin{proof}
			Let $x,y \subseteq D$ and $x\subseteq y$ then $f(x) \subseteq f(y)$.
			
			Let $e \in f(x) \cup A$. If $e \in f(x)$ then $e \in f(y)$ and $e \in f(y) \cup A$. If $e \in A$ then $e \in f(y) \cup A$. We find $f^\cup(x) \subseteq f^\cup(y)$.
			
			Let $e \in f(x) \cap A$. We have $e \in f(x)$ and $e \in A$. Therefore $e \in f(y)$ and $e \in f(y) \cap A$. We find $f^\cap(x) \subseteq f^\cap(y)$.
		\end{proof}
	\end{lemma}

	\paragraph{Fixed-point iteration index}
	We introduce the notion of fixed-point \textit{iteration index} to help with the proof.
	
	Consider alternating fixed-point formula
	\[ \nu X_m.\mu X_{m-1}\dots.\nu X_0.f(X_m,X_{m-1},\dots,X_0) \]
	
	Using fixed-point iteration to solve this formula results in a number of intermediate values for the iteration variables $X_m,\dots X_0$. We define an iteration index that, intuitively, indicates where in the iteration process we are. For an alternating fixed-point formula with $m$ fixed-point variables we define an iteration index $\zeta \subseteq \mathbb{N}^m$.
	
	When applying iteration to formula $\nu X_j.f(X)$ we start with some value for $X_j^0$ and calculate $X_j^{i+1} = f(X_j^{i})$. So we get a list of values for $X_j$, however when we have alternating fixed-point formula's we might iterate $X_j$ multiple times but get different lists of values because the values for $X_m,\dots,X_{j-1}$ are different. We use the iteration index to distinguish between these different lists. 
	
	Iteration index $\zeta = (k_m,\dots,k_0)$ indicates where in the iteration process we are. We start at $\zeta = (0,0,\dots,0)$. We first iterate $X_0$, when we calculate $X_0^1$ we are at iteration index $\zeta=(0,0,\dots,1)$, when we calculate $X_0^2$ we are at iteration index $\zeta=(0,0,\dots,2)$ and so on. In general when we calculate a value for $X_j^i$ then $k_j = i$ in $\zeta$. This gives a natural order of indexes:
	\begin{center}
		$(0,\dots,0,0,0)$\\
		$(0,\dots,0,0,1)$\\
		$(0,\dots,0,0,2)$\\
		$\vdots$\\
		$(0,\dots,0,1,0)$\\
		$(0,\dots,0,1,1)$\\
		$(0,\dots,0,1,2)$\\
		$\vdots$
	\end{center}
	Formally we have $(k_{d-1},\dots,k_0) < (j_{d-1},\dots,j_0)$ if and only if for the largest $l \leq d-1$ such that $k_l \neq j_l$ we have $k_l < j_l$. We define $\{k_{d-1},\dots,k_0\} -1 = \{k_{d-1},\dots,k_0-1\}$ and $\{k_{d-1},\dots,k_0\} +1 = \{k_{d-1},\dots,k_0+1\}$ for convenience of notation.
	
	We write $X_j^\zeta$ to indicate the value of variable $X_j$ at moment $\zeta$ of the iteration process. Variable $X_j$ doesn't change values when a variable $X_l$ with $j>l$ changes values, there we have for indexes $\zeta = (k_m,\dots,k_j,k_{j-1},\dots,k_0)$ and $\zeta' = (k_m,\dots,k_j,k'_{j-1},\dots,k'_0)$ that $X_j^\zeta = X_j^{\zeta'}$.
	
	We can use the fixed-point iteration definition to define the values for $X_j^\zeta$. Let  $\zeta= (k_m,\dots,k_0)$, we have:
	\[ X_0^{\zeta+1} = f(X_m^\zeta, X_{m-1}^\zeta,\dots,X_0^\zeta) \]
	and for any even $0 < j \leq m$
	\[ X_j^{(\dots,k_j+1,\dots)} = \mu X_{j-1}\dots = \bigcup_{i \geq 0}X^{(\dots,k_j,i,\dots)} \]
	and for any odd $0 < j \leq m$
	\[ X_j^{(\dots,k_j+1,\dots)} = \nu X_{j-1}\dots = \bigcap_{i \geq 0}X^{(\dots,k_j,i,\dots)} \]
	
	\paragraph{$\Gamma$-game} We define $\Gamma$, which transforms a parity game, to help with the proof.
	
	Given parity game $G=(V,V_0,V_1,E,\Omega)$ with winning set $W_0$ such that $P_0 \subseteq W_0 \subseteq V \backslash P_1$. We define $\Gamma(G,P_0,P_1) = (V',V_0',V_1',E',\Omega')$ such that
		\begin{align*}
		&V' = (V \backslash P_0 \backslash P_1) \cup \{s_0,s_1\}\\
		&V_0' = (V_0 \cap V') \cup \{s_1\}\\
		&V_1' = (V_1 \cap V') \cup \{s_0\}\\
		&E' = (E \cap (V' \times V')) \cup \{ (v,s_\alpha)\ |\ (v,w) \in E \wedge w \in P_\alpha \}\\
		&\Omega'(v) = \begin{cases}0 & \text{if } v\in \{s_0,s_1\}\\
		\Omega(v) & \text{otherwise}\end{cases}
		\end{align*}
	Parity game $\Gamma(G,P_0,P_1)$ contains vertices $s_0$ and $s_1$ such that they have no outgoing edges and $s_\alpha$ is owned by player $s_{\overline{\alpha}}$. Clearly if the token ends in $s_\alpha$ then player $\alpha$ wins. Vertices that had edges to a vertex in $P_\alpha$ now have an edge to $s_\alpha$.
	
	Note that this parity is not total, as shown in [Monadic second-order logic on tree-like structures by
	Igor Walukiewicz] the formula $S(G)$ also solves non-total games.
	\begin{lemma}
		\label{lem_gamma_same_winner}
		Given parity game $G=(V,V_0,V_1,E,\Omega)$ with winning set $W_0$ such that $P_0 \subseteq W_0 \subseteq V \backslash P_1$ and parity game $G' = \Gamma(G,P_0,P_1)$ with winning set $Q_0$ we have $W_0 \cap (V\backslash P_0 \backslash P_1) = Q_0 \cap (V\backslash P_0 \backslash P_1)$.
		\begin{proof}
			Let vertex $v\in V \backslash P_0 \backslash P_1$. Assume $v$ is won by player $\alpha$ in $G$ using strategy $\sigma_\alpha : V_\alpha \rightarrow V$. We construct a strategy $\sigma'_\alpha :  V'_\alpha \rightarrow V'$ for game $G'$ as follows:
			\[ \sigma'_\alpha(w) = \begin{cases} s_\beta & \text{if $\sigma_\alpha(w) \in P_\beta$ for some $\beta \in \{0,1\}$}\\
			\sigma_\alpha(w) & \text{otherwise}
			\end{cases} \]
			This strategy maps the vertices to the same successors except when a vertex is mapped to a vertex in $P_\beta$, in which case $\sigma'_\alpha$ maps the vertex to $s_\beta$.
			
			Let $\pi'$ be a valid path in $G'$, starting from $v$ and conforming to $\sigma'_\alpha$. Since vertices $s_0$ and $s_1$ don't have any successors we distinguish three cases for $\pi'$:
			\begin{itemize}
				\item Assume $\pi'$ ends in $s_{\overline{\alpha}}$. Let $\pi' = (x_0\dots x_m s_{\overline{\alpha}})$ with $v = x_0$. Because $s_0$ and $s_1$ don't have successors we find $x_i \in V\backslash P_0 \backslash P_1$. Moreover for every $x_ix_{i+1}$ we have $(x_i,x_{i+1}) \in E'$, any such edge is also in $E$ because the edges between vertices in $V\backslash P_0 \backslash P_1$ were left intact when creating $G'$. Finally we find that $(x_m,y) \in E$ with $y \in P_{\overline{\alpha}}$. There must exist a valid path $\pi = (x_0 \dots x_m y\dots)$ in game $G$. Moreover this path conforms to $\sigma_\alpha$ because $\sigma'_\alpha$ and $\sigma_\alpha$ map to the same vertices for all $x_0\dots x_{m-1}$ and $x_m$ maps to a vertex in $P_{\overline{\alpha}}$. Player $\overline{\alpha}$ has a winning strategy from $y$ so we conclude that $\pi$ is won by $\overline{\alpha}$ in game $G$. Because $\pi$ exists and conforms to $\sigma_\alpha$ we find that $\sigma_\alpha$ is not winning for $\alpha$ from $v$ in $G$. This is a contradiction so we conclude that $\pi'$ never ends in $s_{\overline{\alpha}}$.
				\item Assume $\pi'$ ends in $s_\alpha$. In this case player $\alpha$ wins the path.
				\item Assume $\pi'$ never visits $s_\alpha$ or $s_{\overline{\alpha}}$. Assume the path is on by player $\overline{\alpha}$, as we argued above we find that this path is also valid in game $G$, conforms to $\sigma_\alpha$ and is winning for $\overline{\alpha}$. Therefore $\sigma_\alpha$ is not winning for player $\alpha$ from $v$, this is a contradiction so we conclude that player ${\alpha}$ wins the path $\pi'$.
			\end{itemize}
			We find that $\pi'$ is always won by player $\alpha$ in game $G'$. We conclude that any vertex $v \in V \backslash P_0 \backslash P_1$ has the same winner in game $G$ as in game $G'$.
		\end{proof}
	\end{lemma}
\paragraph{Proof}
	\begin{theorem}
		Given parity game $G = (V,V_0,V_1,E,\Omega)$ with winning set $W_0$ such that $P_0\subseteq W_0 \subseteq V\backslash P_1$. The formula 
		\[ S^P(G) = \nu Z_{d-1}.\mu Z_{d-2}\dots \nu Z_0.(G,F_0(Z_{d-1},\dots,Z_0) \cap (V\backslash P_1) \cup P_0) \]
		correctly solves $W_0$ for $G$.
		\begin{proof}
			Let $G' = \Gamma(G,P_0,P_1)$. We consider $S(G')$, which calculates the winning set for player $0$ for game $G'$. Formula $F_0(G',Z_{d-1},\dots,Z_0)$ will always include $s_0$ and never include $s_1$ regardless of the values for $Z_{d-1} \dots Z_0$. Clearly any $\nu Z_i\dots$ or $\mu Z_i\dots$ contains $s_0$ and doesn't contain $s_1$. So we can calculate $S(G')$ using fixed-point iteration starting greatest fixed-point variables at $V'\backslash \{s_1\}$ and least fixed-point variables at $\{s_0\}$.
			
			We can also calculate $S^P(G)$ using fixed-point iteration starting at $P_0$ and $V\backslash P_1$ because clearly any $\nu Z_i \dots$ or $\mu Z_i\dots$ contains all vertices from $P_0$ and none from $P_1$.
			
			We will go through the iteration of formula's $S^P(G)$ and $S(G')$ using iteration index $\zeta$ to indicate where in the iteration we are. We write $Z_i$ to denote variables in $S(G')$ and $Y_i$ to denote variables in $S^P(G)$. 
			
			Trivially, for any $\zeta$ and $i$ we have $P_0 \subseteq Y_i^{\zeta} \subseteq V\backslash P_1$ and $\{s_0\} \subseteq Z_i^\zeta\subseteq V\backslash \{s_1\}$
			
			
			We define operator $\simeq : V \times V' \rightarrow \mathbb{B}$ such that for $Y \subseteq V$ and $Z \subseteq V'$ we have $Y \simeq Z$ if and only if:
			\[ Y \backslash P_0 \backslash P_1 = Z \backslash \{s_0,s_1\}\]
			
			We will prove that for any $\zeta = (k_{d-1},\dots,k_0)$ we have $Y_i^{\zeta} \simeq Z_i^{\zeta}$ for every $i \in [0,d-1]$.
			
			Proof by induction on $\zeta$.
			
			\textbf{Base} $\zeta = (0,0,\dots,0)$: we have for least fixed-point variables $Z_i^\zeta$ and $Y_i^{\zeta}$ the values $\{s_0\}$ and $P_0$, clearly $Z_i^\zeta \simeq Y_i^{\zeta}$. 
			
			For greatest fixed-point variables $Z_j^\zeta$ and $Y_j^{\zeta}$ we have $Z_j^\zeta \backslash \{s_0,s_1\} = V \backslash P_1 \backslash P_0$. So we find $Z_j^\zeta \simeq Y_j^{\zeta}$.
			
			\textbf{Step}: Consider $\zeta = (k_{d-1},\dots,k_0)$. Let $0 \leq j \leq d-1$. If $k_j = 0$ then $Z_j^\zeta = Z_j^{(0,0,\dots,0)}$ and $Y_j^{\zeta} = Y_j^{(0,0,\dots,0)}$, furthermore $Z_j^{(0,0,\dots,0)}\simeq Y_j^{(0,0,\dots,0)}$ so we find $Z_j^{\zeta} \simeq Y_j^{\zeta}$.
			If $k_j > 0$ then we distinguish three cases for $j$ to show that $Z_j^{\zeta} \simeq Y_j^{\zeta}$:
			\begin{itemize}
				\item Case: $j=0$. We have the following equations:
				\[ Y_0^{\zeta} = F_0(G,\tilde{Z}_{d-1}^{\zeta-1},\dots,Y_0^{\zeta-1}) \cap (V\backslash P_1) \cup P_0 \]
				and
				\[ Z_0^{\zeta} = F_0(G',Z_{d-1}^{\zeta-1},\dots,Z_0^{\zeta-1}) \]
				Consider vertex $v \in V\backslash P_0 \backslash P_1$. We distinguish two cases:
				\begin{itemize}
					\item Assume $v \in V_0$.
					
					If $v \in Y_0^{\zeta}$ then $v$ must have an edge in game $G$ to $w$ such that $w\in Y^{\zeta-1}_{\Omega(w)}$. We find $w \notin P_1$ because vertices from $P_1$ are never in the iteration variable. If $w \in P_0$ then it follows from the way we created $G'$ that in $G'$ there exists an edge from $v$ to $s_0$ and since $s_0$ is always in the iteration variable we find $v \in Z_0^{\zeta}$. If $w \notin P_0$ then because $Y^{\zeta-1}_{\Omega(w)} \simeq Z^{\zeta-1}_{\Omega(w)}$ we find $w \in Z^{\zeta-1}_{\Omega(w)}$ and therefore $v \in Z_0^{\zeta}$.
					
					If $v \in Z_0^{\zeta}$ then $v$ must have an edge in game $G'$ to $w$ such that $w\in Z^{\zeta-1}_{\Omega(w)}$. We find $w \neq s_1$ because $w$ is never in the iteration variable. If $w = s_0$ then it follows from the way we created $G'$ that in $G$ there exists an edge from $v$ to a vertex in $P_0$ and since any vertex in $P_0$ is always in the iteration variable we find $v \in Y_0^{\zeta}$. If $w \neq s_0$ then because $Y^{\zeta-1}_{\Omega(w)} \simeq Z^{\zeta-1}_{\Omega(w)}$ we find $w \in Y^{\zeta-1}_{\Omega(w)}$ and therefore $v \in Y_0^{\zeta}$.
					\item Assume $v \in V_1$.
					
					If $v \in Y_0^{\zeta}$ then for any successor $w$ of $v$ in game $G$ it holds that $w \in Y^{\zeta-1}_{\Omega(w)}$. Consider successor $x$ of $v$ in game $G'$. We distinguish three cases:
					\begin{itemize}
						\item $x = s_0$: In this case $x \in Z^{\zeta-1}_{\Omega(x)}$ because $s_0$ is always in the iteration variables.
						\item $x = s_1$: Because of the way $G'$ is constructed we find vertex $v$ must have a successor $w$ in $P_1$, however we found $w \in Y^{\zeta-1}_{\Omega(w)}$. This is a contradiction because vertices in $P_1$ are never in the iteration variables. So this case can not happen.
						\item $x \notin \{s_0,s_1\}$: We have $x \in V'\backslash \{s_0,s_1\}$ and therefore $x$ is also a successor of $v$ in game $G$. We find $x \in Y^{\zeta-1}_{\Omega(x)}$ and because $Y^{\zeta-1}_{\Omega(x)} \simeq Z^{\zeta-1}_{\Omega(x)}$ we have $x \in Z^{\zeta-1}_{\Omega(x)}$.
					\end{itemize}
					We always find  $x \in Z^{\zeta-1}_{\Omega(x)}$, therefore $v \in Z_0^{\zeta}$.
					
					If $v \in Z_0^{\zeta}$ then for any successor $w$ of $v$ in game $G'$ it holds that $w \in Z^{\zeta-1}_{\Omega(w)}$. Consider successor $x$ of $v$ in game $G$. We distinguish three cases:
					\begin{itemize}
						\item $x \in P_0$: In this case $x \in Y^{\zeta-1}_{\Omega(x)}$ because vertices in $P_0$ are always in the iteration variables.
						\item $x \in P_1$: Because of the way $G'$ is constructed we find vertex $v$ must have successor $s_1$ in game $G'$, however we found that for any successor $w$ of $v$ in game $G'$ we have $w \in Z^{\zeta-1}_{\Omega(w)}$. This is a contradiction because $s_1$ is never in the iteration variable. So this case can not happen.
						\item $x \in V \backslash P_0 \backslash P_1$: We find that $x$ is also a successor of $v$ in game $G'$. We find $x \in Z^{\zeta-1}_{\Omega(w)}$ and because $Y^{\zeta-1}_{\Omega(x)} \simeq Z^{\zeta}_{\Omega(x)}$ we have $x \in Y^{\zeta}_{\Omega(x)}$.
					\end{itemize}
					We always find  $x \in Y^{\zeta-1}_{\Omega(x)}$, therefore $v \in Y_0^{\zeta}$.
				\end{itemize}
			
			\item Case: $j > 0$ being even. We have 
			\[ Z_j^{\zeta} = \mu Z_{j-1}\dots = \bigcup_{i\geq 0} Z_{j-1}^{\{k_{d-1},\dots,k_j-1,i,\dots\}}\]
			and 
			
			\[ Y_j^{\zeta} = \mu Y_{j-1}\dots = \bigcup_{i\geq 0} Y_{j-1}^{\{k_{d-1},\dots,k_j-1,i,\dots\}}\]
			
			Let $v \in V \backslash P_0 \backslash P_1$.
			
			If $v \in Z_j^{\zeta}$ then there exists some $i$ such that $v \in Z_{j-1}^{\{k_{d-1},\dots,k_j-1,i,\dots\}}$. Since $\{k_{d-1},\dots,k_j-1,i,\dots\} < \zeta$ we apply induction to find $Z_{j-1}^{\{k_{d-1},\dots,k_j-1,i,\dots\}} \simeq Y_{j-1}^{\{k_{d-1},\dots,k_j-1,i,\dots\}}$. Because $v \in V \backslash P_0 \backslash P_1$ we find $v \in Y_{j-1}^{\{k_{d-1},\dots,k_j-1,i,\dots\}}$ and therefore $v \in Y_j^{\zeta}$.
			
			If $v \in Y_j^{\zeta}$ then we apply symmetrical reasoning to find $v \in Z_j^{\zeta}$.
			\item Case: $j > 0$ being odd. We have 
			
			\[ Z_j^{\zeta} = \nu Z_{j-1}\dots = \bigcap_{i\geq 0} Z_{j-1}^{\{k_{d-1},\dots,k_j-1,i,\dots\}}\]
			and 
			
			\[ Y_j^{\zeta} = \nu Y_{j-1}\dots = \bigcap_{i\geq 0} Y_{j-1}^{\{k_{d-1},\dots,k_j-1,i,\dots\}}\]
			
			Let $v \in V \backslash P_0 \backslash P_1$.
			
			If $v \in Z_j^{\zeta}$ then for all $i \geq 0$ we have $v \in Z_{j-1}^{\{k_{d-1},\dots,k_j-1,i,\dots\}}$. Assume $v \notin Y_j^{\zeta}$, there must exist an $l \geq 0$ such that $v \notin Y_j^{\{k_{d-1},\dots,k_j-1,l,\dots\}}$. Since $\{k_{d-1},\dots,k_j-1,l,\dots\} < \zeta$ we apply induction to find $Z_{j-1}^{\{k_{d-1},\dots,k_j-1,l,\dots\}} \simeq Y_{j-1}^{\{k_{d-1},\dots,k_j-1,l,\dots\}}$. Because $v \in V \backslash P_0 \backslash P_1$ we find $v \notin Z_{j-1}^{\{k_{d-1},\dots,k_j-1,i,\dots\}}$ which is a contradiction so we have $v \in Y_j^{\zeta}$.
			
			If $v \in Y_j^{\zeta}$ then we apply symmetrical reasoning to find $v \in Z_j^{\zeta}$.
			\end{itemize}
			
			This proves that for any $\zeta$ we have $Y_i^{\zeta} \simeq Z_i^{\zeta}$ for every $i \in [0,d-1]$.
			
			We have shown that when starting the iteration of $S(G')$ and $S^P(G)$ at specific values then we get a similar results for vertices in $V \backslash P_0 \backslash P_1$. We choose these values such that they solve the formula's correctly, so we conclude that $S(G') \backslash \{s_0,s_1\} = S^P(G) \backslash P_0 \backslash P_1$. Lemma \ref{lem_gamma_same_winner} shows that $S(G')$ correctly vertices in $V \backslash P_0 \backslash P_1$ for game $G$. So $S^P(G)$ also correctly solves vertices $V \backslash P_0 \backslash P_1$ for game $G$. 
			
			Moreover any vertex in $P_0$ is in $S^P(G)$, which is correct because $P_0$ vertices are winning for player 0. Any vertex in $P_1$ is not in $S^P(G)$, which is correct because $P_1$ vertices are winning for player 1. We conclude that all vertices are correctly solved.
		\end{proof}
	\end{theorem}
	This gives the following algorithm where we start iteration at $P_0$ and $V\backslash P_1$. Moreover we ignore vertices in $P_0$ or $P_1$ in the diamond and box calculation, finally we always add vertices in $P_0$ to the results of the diamond and box operator.
	
	\begin{algorithm}
		\caption{Fixed-point iteration with $P_0$ and $P_1$}
		\label{alg_FPITE}
		\begin{multicols}{2}
			\begin{algorithmic}[1]
				\Function{FPIter}{$G = (V, V_0, V_1, E, \Omega), P_0, P_1$}
				\For{$i \gets d-1,\dots,0$}
				\State $\textsc{Init}(i)$
				\EndFor
				\Repeat
				\State $Z_0'\gets Z_0$
				\State $Z_0 \gets P_0 \cup \textsc{Diamond}() \cup \textsc{Box}()$
				\State $i \gets 0$
				\While{$Z_i=Z_i' \wedge i < d-1$}
				\State $i \gets i+1$
				\State $Z_i' \gets Z_i$
				\State $Z_i \gets Z_{i-1}$
				\State $\textsc{Init}(i-1)$
				\EndWhile
				\Until{$i = d-1 \wedge Z_{d-1} = Z_{d-1}'$}
				\State \Return $(Z_{d-1},V\backslash Z_{d-1})$
				\EndFunction
			\end{algorithmic}\bigskip\bigskip
			\begin{algorithmic}[1]
				\Function{Init}{$i$}
				\State $Z_i \gets P_0$ if $i$ is odd, $V\backslash P_1$ otherwise
				\EndFunction
			\end{algorithmic}\bigskip
			\begin{algorithmic}[1]
				\Function{Diamond}{}
				\State \Return $\{ v \in V_0\backslash P_0 \backslash P_1\ |\ \exists_{w\in V} (v,w) \in E \wedge w \in Z_{\Omega(w)}\}$
				\EndFunction
			\end{algorithmic}\bigskip
			\begin{algorithmic}[1]
				\Function{Box}{}
				\State \Return $\{ v \in V_1\backslash P_0 \backslash P_1\ |\ \forall_{w\in V} (v,w) \in E \implies w \in Z_{\Omega(w)}\}$
				\EndFunction
			\end{algorithmic}
		\end{multicols}
	\end{algorithm}

Note: In the paper [The Fixpoint-Iteration Algorithm for Parity Games], that describes an algorithm to solve $S(G)$, total parity games are used. However the argumentation only relies on the fact that every vertex has a unique owner and priority. So the algorithm can also solve non total games.
\end{document}