We first look at labelled transition systems (LTSs) and the modal $\mu$-calculus and what it means to verify an LTS. The definition below are derived from \cite{Groote}.
\begin{definition}
	\label{def_lts}A labelled transition system (LTS) is a tuple $M = (S, Act, trans, s_0)$, where:
	\begin{itemize}
		\item $S$ is a set of states,
		\item $Act$ a set of actions,
		\item $trans \subseteq S \times Act \times S$ is the transition relation with $(s,a,s') \in trans$ denoted by $s \xrightarrow a s'$,
		\item $s_0 \in S$ is the initial state.
	\end{itemize}
\end{definition}
Consider the example in figure \ref{fig:coffeemachinebasiceurolts} (directly taken from \cite{FamBasedModelCheckingWithMCRL2}) of a coffee machine where we have two actions: ins (insert coin) and std (get standard sized coffee).\\
\begin{figure}[h]
	\centering
	\includegraphics[scale=0.5]{Examples/CoffeeMachine/BasicEuroLTS}
	\caption[Coffee machine LTS]{Coffee machine LTS $C$}
	\label{fig:coffeemachinebasiceurolts}
\end{figure}


\begin{definition}
	\label{def_mu_syntax}
	A modal $\mu$-calculus formula over the set of actions $Act$ and a set of variables $\mathcal{X}$ is defined by
	\[ \varphi = \top\ |\ \bot\ |\ X\ |\ \varphi \vee \varphi\ |\ \varphi \wedge \varphi\ |\ \langle a \rangle \varphi\ |\ [a]\varphi\ |\ \mu X.\varphi\ |\ \nu X.\varphi \]
	with $a \in Act$ and $X \in \mathcal{X}$. 
	
	
	No negations in the language because negations can be pushed inside to the propositions, ie. the $\top$ and $\bot$ elements.
\end{definition}
The modal $\mu$-calculus contains boolean constants $\top$ and $\bot$, propositional operators $\vee$ and $\wedge$, modal operators $\langle \, \rangle$ and $[ \, ]$ and fixpoint operators $\mu$ and $\nu$. A formula is closed when variables only occur in the scope of a fixpoint operator for that variable.

A modal $\mu$-calculus formula can be interpreted with an LTS, this results in a set of states for which the formula holds.
\begin{definition}
	\label{def_mu_sem} For LTS $(S, Act, trans, s_0)$ we inductively define the interpretation of a modal $\mu$-calculus formula $\varphi$, notation
	$[\![ \varphi ]\!]^\eta$, where $\eta : \mathcal{X} \rightarrow \mathcal{P}(S)$ is a logical variable valuation, as a set of states
	where $\varphi$ is valid, by:
	\begin{align*}
	&[\![ \mathit{\top} ]\!]^\eta &&= S\\
	&[\![ \mathit{\bot} ]\!]^\eta &&= \emptyset\\
	&[\![ \varphi_1 \wedge \varphi_2 ]\!]^\eta &&= [\![ \varphi_1 ]\!]^\eta \cap [\![ \varphi_2 ]\!]^\eta \\
	&[\![ \varphi_1 \vee \varphi_2 ]\!]^\eta &&= [\![ \varphi_1 ]\!]^\eta \cup [\![ \varphi_2 ]\!]^\eta\\
	&[\![ \langle a \rangle \varphi ]\!]^\eta &&= \{s \in S|\exists_{s' \in S} s \xrightarrow {a} s' \wedge s' \in [\![ \varphi ]\!]^\eta\}\\
	&[\![ [ a ] \varphi ]\!]^\eta &&= \{s \in S|\forall_{s' \in S} s \xrightarrow {a} s' \implies s' \in [\![ \varphi ]\!]^\eta\}\\
	&[\![ \mu X. \varphi ]\!]^\eta &&= \bigcap_{f \subseteq S}\{f | f = [\![ \varphi ]\!]^{\eta[X:=f]}\}\\
	&[\![ \nu X. \varphi ]\!]^\eta &&= \bigcup_{f \subseteq S}\{f | f = [\![ \varphi ]\!]^{\eta[X:=f]}\}\\
	&[\![ X ]\!]^\eta &&= \eta(X)
	\end{align*}
\end{definition}

Given closed formula $\varphi$, LTS $M = (S, Act, trans, s_0)$ and $s \in S$ we write $(M,s) \models \varphi$ iff $s \in [\![ \varphi ]\!]^\eta$ for $M$, we say that formula $\varphi$ holds for $M$ in state $s$. If formula $\varphi$ holds for $M$ in the initial state we say that formula $\varphi$ holds for $M$ and write $M \models \varphi$.

Again consider the coffee machine example (figure \ref{fig:coffeemachinebasiceurolts}) and formula $\varphi = \nu X. \mu Y. ([ins]Y \wedge [std] X)$ (taken from \cite{FamBasedModelCheckingWithMCRL2}) which states that action std must occur infinitely often over all runs. Obviously this holds for the coffee machine, therefore we have $C \models \varphi$.

\subsection{Featured transition systems}
A featured transition system extends the LTS definition to express variability. It does so by introducing \textit{features} and \textit{products} into the definition. Features are options that can be enabled or disabled for the system. A product is a feature assignments, ie. a set of features that is enabled for that product. Not a products are valid, some features might be mutually exclusive and some features might always be required. To express products one can use feature diagrams as explained in \cite{Classen2013FeaturedTS}. Feature diagrams offer a nice way of expressing which feature assignments are valid, but since they offer just that we simply represent the valid products with a set of feature assignments. Finally FTSs guards every transition with a boolean expression over the set of features. We have the following definition, based on \cite{Classen2013FeaturedTS}:
\begin{definition}
	\label{def_fts}A featured transition system (FTS) is a tuple $M = (S, Act, trans, s_0, N, P, \gamma)$, where:
	\begin{itemize}
		\item $S, Act, trans, s_0$ are defined as in an LTS,
		\item $N$ is a non-empty set of features,
		\item $P \subseteq \mathcal{P}(N)$ is a set of products, ie. feature assignments, that are valid,
		\item $\gamma : trans \rightarrow \mathbb{B}(N)$ is a total function, labelling each transition with a boolean expression over the features. A product $p \in \mathcal{P}(N)$ satisfying the boolean expression of transition $t$ is denoted by $p \models \gamma(t)$, $\gamma(t)(p) = 1$ or $p \in [\![\gamma(t)]\!]$. The boolean expression that is satisfied by any feature assignment is denoted by $\top$, ie $p \models \top$ for any $p$.
		
		A transition $s \xrightarrow a s'$ and $\gamma((s,a,s')) = f$ is denoted by $s \xrightarrow {a | f} s'$. 
	\end{itemize}
\end{definition}
\begin{figure}[h]
\centering
\includegraphics[scale=0.5]{Examples/CoffeeMachine/FTS}
\caption[Coffee machine LTS]{Coffee machine FTS $C$}
\label{fig:coffeemachinefts}
\end{figure}

Consider the example in figure \ref{fig:coffeemachinefts} (directly taken from \cite{FamBasedModelCheckingWithMCRL2}) which shows an FTS for a coffee machine For this example we have two features $N = \{\$, \officialeuro\}$ and two valid products $P = \{\{\$\},\{\officialeuro\}\}$.

An FTS expresses they behaviour of multiple products, we can derive the behaviour of a single product by simply removing all the transitions from the FTS for which the product doesn't satisfy the feature expression guarding the transition. We call this a \textit{projection} (\cite{Classen2013FeaturedTS}).

\begin{definition}
	\label{def_fts_proj}
	The projection of an FTS $M = (S, Act, trans, s_0, N, P, \gamma)$ to a product $p \in P$, noted $M_{|p}$, is the LTS $M'=(S,Act,trans', s_0)$, where $trans' = \{t \in trans\ |\ p \models \gamma(t)\}$.
\end{definition}
The coffee machine example can be projected to its two products, which results in the LTSs in figure \ref{fig:cofeemachineftsproj}.
\begin{figure}[h]
	\centering
	\begin{subfigure}{.5\textwidth}
		\centering
		\includegraphics[width=1\linewidth]{Examples/CoffeeMachine/FTSProjDollar}
		\caption[$C_{|\{\$\}}$]{$C$ projected to the dollar product: $C_{|\{\$\}}$}
		\label{fig:coffeemachineftsprojdollar}
	\end{subfigure}%
	\begin{subfigure}{.5\textwidth}
		\centering
		\includegraphics[width=1\linewidth]{Examples/CoffeeMachine/FTSProjEuro}
		\caption[$C_{|\{\$\}}$]{$C$ projected to the euro product: $C_{|\{\officialeuro\}}$}
		\label{fig:coffeemachineftsprojeuro}
	\end{subfigure}
	\caption{Projections of the coffee machine FTS}
	\label{fig:cofeemachineftsproj}
\end{figure}

We want to verify the FTS against a modal $\mu$-calculus formula $\varphi$. That is, we want to find out for which products in the FTS its projection satisfies $\varphi$. Formally, given FTS $M = (S, Act, trans, s_0, N, P, \gamma)$ and modal $\mu$-calculus formula $\varphi$ we want to find $P_s \subseteq P$ such that:
\begin{itemize}
	\item for every $p \in P_s$ we have $M_{|p} \models \varphi$ and
	\item for every $p \in P \backslash P_s$ we have $M_{|p} \not\models \varphi$.
\end{itemize}
Furthermore a counterexample for every $p \in P \backslash P_s$ is preferred.