\documentclass[aspectratio=169]{beamer}
\setbeamercovered{dynamic}

\usetheme{default}

\usepackage{ wasysym }
\usepackage{geometry}
\usepackage{amsthm}
\usepackage{xcolor}
\usepackage{mathtools}
\usepackage{amsfonts}
\usepackage{multicol}
\usepackage{graphicx} % Allows including images
\usepackage{booktabs} % Allows the use of \toprule, \midrule and \bottomrule in tables
\usepackage{tikz,pgfplots}
\tikzset{>=latex}
\usetikzlibrary{arrows,automata,shapes,calc,matrix}
\definecolor{bgTUE}{HTML}{f1efef}
\definecolor{fgTUE}{HTML}{000000}
\definecolor{tokencolor}{HTML}{6792d6}
\definecolor{highlightgraph}{HTML}{aedb81}
%----------------------------------------------------------------------------------------
%	TITLE PAGE
%----------------------------------------------------------------------------------------

\title{Verifying SPLs using parity games expressing variability} % The short title appears at the bottom of every slide, the full title is only on the title page
\newcounter{picite}
\newcounter{piciteb}

\institute[TU/e] % Your institution as it will appear on the bottom of every slide, may be shorthand to save space
{
	\begin{columns}
\begin{column}{0.5\textwidth}
	\begin{center}
	\begin{tikzpicture}[->,scale=0.7, every node/.style={scale=0.8}]
	\tikzstyle{even} = [diamond,draw,minimum size=0.75cm]
	\tikzstyle{odd}  = [rectangle,draw,shape aspect=1,minimum size=0.6cm]
	
	\node[odd, label=north:$v_1$,fill=teal,text=white] (v1) at (20,20) {5};
	\node[even,label=north:$v_2$,fill=teal,text=white] (v2) at (23,20) {6};
	\node[even,label=north:$v_3$,fill=teal,text=white] (v3) at (26,20) {4};
	
	\node[even,label=west:$v_4$,fill=teal,text=white]  (v4) at (20,17) {1};
	\node[odd, label=north east:$v_5$,fill=olive,text=white] (v5) at (23,17) {2};
	\node[odd, label=east:$v_6$,fill=pink] (v6) at (26,17) {3};
	
	\node[odd, label=south:$v_7$,fill=pink] (v7) at (23,14) {2};
	
	
	\path (v1) edge node[above]{$\{c_1,c_2,c_3\}$} (v2);
	\path (v3) edge node[above]{$\{c_1,c_2,c_3\}$} (v2);
	
	\path (v4) edge node[left]{$\{c_1,c_2\}$} (v1);
	\path (v2) edge node[fill=bgTUE,near start]{$\{c_1,c_2\}$} (v4);
	\path (v4) edge node[above]{$\{c_1,c_3\}$} (v5);
	\path (v5) edge node[fill=bgTUE]{$\{c_1,c_2,c_3\}$} (v2);
	\path (v6) edge node[above]{$\{c_1,c_2\}$} (v5);
	\path (v2) edge node[fill=bgTUE,near start]{$\{c_1,c_3\}$} (v6);
	\path (v6) edge node[right]{$\{c_1,c_3\}$} (v3);
	
	
	\path (v7) edge node[fill=bgTUE]{$\{c_1,c_2,c_3\}$} (v4);
	\path (v5) edge node[left,near start]{$\{c_1,c_3\}$} (v7);
	\path (v7) edge[bend left] node[fill=bgTUE]{$\{c_1,c_2\}$} (v6);
	\path (v6) edge[bend left] node[fill=bgTUE]{$\{c_1,c_2\}$} (v7);
	\end{tikzpicture}
\end{center}
\end{column}
	\begin{column}{0.5\textwidth}
		\begin{center}
			\large
			Sjef van Loo\\
			\medskip\small
			6 November, 2019\\
			\bigskip
			\bigskip
			\bigskip
			\bigskip
			\bigskip
			\bigskip
\textit{Msc Thesis \\
Computer Science and Engineering\\
\medskip
Supervised by T.A.C. Willemse}\\
\end{center}
\end{column}
\end{columns}
}
\date{November 6, 2019} % Date, can be changed to a custom date
\setbeamertemplate{footline}[frame number]
\begin{document}
	\setbeamercolor{footline}{bg=white}
\setbeamercolor{normal text}{fg=fgTUE,bg=bgTUE}\usebeamercolor*{normal text}
\setbeamercolor{title}{fg=fgTUE}
\setbeamercolor{subtitle}{fg=fgTUE}
\setbeamercolor{titlelike}{fg=fgTUE}
\setbeamercolor{alerted text}{fg=fgTUE}
\setbeamercolor{example text}{fg=fgTUE}

%\setbeamercolor{structure}{fg=fgTUE}

\setbeamercolor{background canvas}{parent=normal text}
%\setbeamercolor{background}{parent=background canvas}

\setbeamercolor{palette primary}{fg=fgTUE,bg=bgTUE} % changed this
\setbeamercolor{palette secondary}{use=structure,fg=fgTUE} % changed this
\setbeamercolor{palette tertiary}{use=structure,fg=fgTUE} % changed this
\setbeamercolor{box1}{fg=fgTUE,bg=white}
\setbeamercolor{itemize item}{fg=fgTUE,bg=white}
\setbeamercolor{itemize subitem}{fg=fgTUE,bg=white}
\setbeamercolor{itemize subsubitem}{fg=fgTUE,bg=white}
\setbeamercolor{enumerate item}{fg=fgTUE,bg=white}
\setbeamercolor{enumerate subitem}{fg=fgTUE,bg=white}
\setbeamercolor{enumerate subsubitem}{fg=fgTUE,bg=white}
\setbeamertemplate{footline}{
	{%
		\leavevmode%
		\hbox{%
			\begin{beamercolorbox}[wd=.1\paperwidth,ht=3ex,dp=1ex,right]{box1}%
				\insertframenumber
			\end{beamercolorbox}%
			\begin{beamercolorbox}[wd=0.05\paperwidth,ht=3ex,dp=1ex,left]{box1}%
			\end{beamercolorbox}%
			\begin{beamercolorbox}[wd=0.7\paperwidth,ht=3ex,dp=1ex,left]{box1}%
				\inserttitle
			\end{beamercolorbox}%
			\begin{beamercolorbox}[wd=0.12\paperwidth,ht=3ex,dp=1ex,right]{box1}%
				\includegraphics[height=3ex]{TUE}
			\end{beamercolorbox}%
			\begin{beamercolorbox}[wd=0.03\paperwidth,ht=3ex,dp=1ex,right]{box1}%
			\end{beamercolorbox}}%
}}

\begin{frame}
\titlepage % Print the title page as the first slide
\end{frame}
%----------------------------------------------------------------------------------------
%	PRESENTATION SLIDES
%----------------------------------------------------------------------------------------

\begin{frame}[t]
\frametitle{Outline}
\begin{columns}[t]
	\begin{column}{0.5\textwidth}
		\begin{itemize}
			\item Verification \& SPLs
			\item Problem statement
			\item Variability Parity Games
			\item VPG algorithms
			\item Experimental results
		\end{itemize}
	\end{column}
	\begin{column}{0.5\textwidth}
	\end{column}
\end{columns}
\end{frame}

%------------------------------------------------

\begin{frame}[t]
\frametitle{Verification}
	\begin{itemize}
		\item Creating correct software is difficult
		\item Even when testing is done rigorously errors can slip in\pause
		\item Mathematically \textit{model the behaviour} of software (LTS)
		\item Mathematically \textit{specify a requirement} (modal $\mu$-calculus)\pause
		\item Check if the model satisfies the requirement
	\end{itemize}
\begin{center}
	\begin{tikzpicture}[->]
	\tikzstyle{even} = [diamond,draw,minimum size=0.75cm]
	\tikzstyle{odd}  = [rectangle,draw,shape aspect=1,minimum size=0.75cm]
	
	\node (lts) at (1,3.5)  {LTS $M$};
	\node (form)at (5,3.5)  {Modal $\mu$-calculus formula $\varphi$};
	\node (check)at (3,1.5){$M$ satisfies $\varphi$?};
	
	\path[-] (lts) edge (3,2.5);
	\path[-] (form)edge (3,2.5);
	\path (3,2.5)  edge (check);
	\end{tikzpicture}
\end{center}
\end{frame}
%------------------------------------------------

\begin{frame}[t]
\frametitle{Software product lines}
\begin{itemize}
	\item Software product lines are configurable systems
	\item Many variants of the same system, i.e. \textit{software products}
	\item FTSs express multiple LTSs using \textit{features}
\end{itemize}
\end{frame}
%------------------------------------------------

\begin{frame}[t]
\frametitle{Problem statement}
	\begin{itemize}
		\item Find all the products in an SPL that satisfy a requirement
	\end{itemize}
	\begin{center}
		\begin{tikzpicture}[->]
		\tikzstyle{even} = [diamond,draw,minimum size=0.75cm]
		\tikzstyle{odd}  = [rectangle,draw,shape aspect=1,minimum size=0.75cm]
		
		\node (lts) at (1,3.5)  {FTS $M$};
		\node (form)at (5,3.5)  {Modal $\mu$-calculus formula $\varphi$};
		\node (check)at (3,1.5) {Which products in $M$ satisfy $\varphi$?};
		
		\path[-] (lts) edge (3,2.5);
		\path[-] (form)edge (3,2.5);
		\path (3,2.5)  edge (check);
		\end{tikzpicture}
	\end{center}
	\begin{itemize}
		\item Do so more efficiently than verifying every product independently
	\end{itemize}
\end{frame}

%----------------------------------------------------------------------------------------

\begin{frame}[t]
\frametitle{Variability parity game}
\begin{columns}[t]
	\begin{column}{0.5\textwidth}
		\large Parity game.\small\\
		Players 0 (even,$\Diamond$) and 1 (odd,$\square$)\\
		\begin{center}
			\setcounter{picite}{0}%
			\def\showprio{0}
			\def\token{0}%
			\stepcounter{picite}\only<\thepicite>{\if\token1\def\fillva{tokencolor}\else\def\fillva{bgTUE}\fi%
\if\token2\def\fillvb{tokencolor}\else\def\fillvb{bgTUE}\fi%
\if\token3\def\fillvc{tokencolor}\else\def\fillvc{bgTUE}\fi%
\if\token4\def\fillvd{tokencolor}\else\def\fillvd{bgTUE}\fi%
\if\token5\def\fillve{tokencolor}\else\def\fillve{bgTUE}\fi%
\begin{tikzpicture}[->,scale=0.9, every node/.style={scale=0.8}]
	\tikzstyle{even} = [diamond,draw,minimum size=0.75cm]
	\tikzstyle{odd}  = [rectangle,draw,shape aspect=1,minimum size=0.6cm]
	
	\node[even, label=west:$v_1$,fill=\fillva,label=center:\if\showprio1{3}\fi] (v1) at (0,2) {};
	\node[even, label=east:$v_2$,fill=\fillvb,label=center:\if\showprio1{0}\fi] (v2) at (2,2) {};
	\node[odd,label=west:$v_3$,fill=\fillvc,label=center:\if\showprio1{2}\fi]  (v3) at (0,0) {};
	\node[even,label=east:$v_4$,fill=\fillvd,label=center:\if\showprio1{2}\fi] (v4) at (2,0) {};
	\node[odd,label=east:$v_5$,fill=\fillve,label=center:\if\showprio1{1}\fi] (v5) at (4,1) {};
	
	\path (v4) edge (v5) ;
	\path (v3) edge [bend right] (v1) ;
	\path (v1) edge [bend right] (v3) ;
	\path (v3) edge (v4) ;
	\path (v2) edge (v4) ;
	\path (v2) edge (v1) ;
	\path (v5) edge (v2) ;
\end{tikzpicture}}\pause%
			\def\token{1}%
			\stepcounter{picite}\only<\thepicite>{\if\token1\def\fillva{tokencolor}\else\def\fillva{bgTUE}\fi%
\if\token2\def\fillvb{tokencolor}\else\def\fillvb{bgTUE}\fi%
\if\token3\def\fillvc{tokencolor}\else\def\fillvc{bgTUE}\fi%
\if\token4\def\fillvd{tokencolor}\else\def\fillvd{bgTUE}\fi%
\if\token5\def\fillve{tokencolor}\else\def\fillve{bgTUE}\fi%
\begin{tikzpicture}[->,scale=0.9, every node/.style={scale=0.8}]
	\tikzstyle{even} = [diamond,draw,minimum size=0.75cm]
	\tikzstyle{odd}  = [rectangle,draw,shape aspect=1,minimum size=0.6cm]
	
	\node[even, label=west:$v_1$,fill=\fillva,label=center:\if\showprio1{3}\fi] (v1) at (0,2) {};
	\node[even, label=east:$v_2$,fill=\fillvb,label=center:\if\showprio1{0}\fi] (v2) at (2,2) {};
	\node[odd,label=west:$v_3$,fill=\fillvc,label=center:\if\showprio1{2}\fi]  (v3) at (0,0) {};
	\node[even,label=east:$v_4$,fill=\fillvd,label=center:\if\showprio1{2}\fi] (v4) at (2,0) {};
	\node[odd,label=east:$v_5$,fill=\fillve,label=center:\if\showprio1{1}\fi] (v5) at (4,1) {};
	
	\path (v4) edge (v5) ;
	\path (v3) edge [bend right] (v1) ;
	\path (v1) edge [bend right] (v3) ;
	\path (v3) edge (v4) ;
	\path (v2) edge (v4) ;
	\path (v2) edge (v1) ;
	\path (v5) edge (v2) ;
\end{tikzpicture}}\pause%
			\def\token{3}%
			\stepcounter{picite}\only<\thepicite>{\if\token1\def\fillva{tokencolor}\else\def\fillva{bgTUE}\fi%
\if\token2\def\fillvb{tokencolor}\else\def\fillvb{bgTUE}\fi%
\if\token3\def\fillvc{tokencolor}\else\def\fillvc{bgTUE}\fi%
\if\token4\def\fillvd{tokencolor}\else\def\fillvd{bgTUE}\fi%
\if\token5\def\fillve{tokencolor}\else\def\fillve{bgTUE}\fi%
\begin{tikzpicture}[->,scale=0.9, every node/.style={scale=0.8}]
	\tikzstyle{even} = [diamond,draw,minimum size=0.75cm]
	\tikzstyle{odd}  = [rectangle,draw,shape aspect=1,minimum size=0.6cm]
	
	\node[even, label=west:$v_1$,fill=\fillva,label=center:\if\showprio1{3}\fi] (v1) at (0,2) {};
	\node[even, label=east:$v_2$,fill=\fillvb,label=center:\if\showprio1{0}\fi] (v2) at (2,2) {};
	\node[odd,label=west:$v_3$,fill=\fillvc,label=center:\if\showprio1{2}\fi]  (v3) at (0,0) {};
	\node[even,label=east:$v_4$,fill=\fillvd,label=center:\if\showprio1{2}\fi] (v4) at (2,0) {};
	\node[odd,label=east:$v_5$,fill=\fillve,label=center:\if\showprio1{1}\fi] (v5) at (4,1) {};
	
	\path (v4) edge (v5) ;
	\path (v3) edge [bend right] (v1) ;
	\path (v1) edge [bend right] (v3) ;
	\path (v3) edge (v4) ;
	\path (v2) edge (v4) ;
	\path (v2) edge (v1) ;
	\path (v5) edge (v2) ;
\end{tikzpicture}}\pause%
			\def\token{4}%
			\stepcounter{picite}\only<\thepicite>{\if\token1\def\fillva{tokencolor}\else\def\fillva{bgTUE}\fi%
\if\token2\def\fillvb{tokencolor}\else\def\fillvb{bgTUE}\fi%
\if\token3\def\fillvc{tokencolor}\else\def\fillvc{bgTUE}\fi%
\if\token4\def\fillvd{tokencolor}\else\def\fillvd{bgTUE}\fi%
\if\token5\def\fillve{tokencolor}\else\def\fillve{bgTUE}\fi%
\begin{tikzpicture}[->,scale=0.9, every node/.style={scale=0.8}]
	\tikzstyle{even} = [diamond,draw,minimum size=0.75cm]
	\tikzstyle{odd}  = [rectangle,draw,shape aspect=1,minimum size=0.6cm]
	
	\node[even, label=west:$v_1$,fill=\fillva,label=center:\if\showprio1{3}\fi] (v1) at (0,2) {};
	\node[even, label=east:$v_2$,fill=\fillvb,label=center:\if\showprio1{0}\fi] (v2) at (2,2) {};
	\node[odd,label=west:$v_3$,fill=\fillvc,label=center:\if\showprio1{2}\fi]  (v3) at (0,0) {};
	\node[even,label=east:$v_4$,fill=\fillvd,label=center:\if\showprio1{2}\fi] (v4) at (2,0) {};
	\node[odd,label=east:$v_5$,fill=\fillve,label=center:\if\showprio1{1}\fi] (v5) at (4,1) {};
	
	\path (v4) edge (v5) ;
	\path (v3) edge [bend right] (v1) ;
	\path (v1) edge [bend right] (v3) ;
	\path (v3) edge (v4) ;
	\path (v2) edge (v4) ;
	\path (v2) edge (v1) ;
	\path (v5) edge (v2) ;
\end{tikzpicture}}\pause%
			\def\token{5}%
			\stepcounter{picite}\only<\thepicite>{\if\token1\def\fillva{tokencolor}\else\def\fillva{bgTUE}\fi%
\if\token2\def\fillvb{tokencolor}\else\def\fillvb{bgTUE}\fi%
\if\token3\def\fillvc{tokencolor}\else\def\fillvc{bgTUE}\fi%
\if\token4\def\fillvd{tokencolor}\else\def\fillvd{bgTUE}\fi%
\if\token5\def\fillve{tokencolor}\else\def\fillve{bgTUE}\fi%
\begin{tikzpicture}[->,scale=0.9, every node/.style={scale=0.8}]
	\tikzstyle{even} = [diamond,draw,minimum size=0.75cm]
	\tikzstyle{odd}  = [rectangle,draw,shape aspect=1,minimum size=0.6cm]
	
	\node[even, label=west:$v_1$,fill=\fillva,label=center:\if\showprio1{3}\fi] (v1) at (0,2) {};
	\node[even, label=east:$v_2$,fill=\fillvb,label=center:\if\showprio1{0}\fi] (v2) at (2,2) {};
	\node[odd,label=west:$v_3$,fill=\fillvc,label=center:\if\showprio1{2}\fi]  (v3) at (0,0) {};
	\node[even,label=east:$v_4$,fill=\fillvd,label=center:\if\showprio1{2}\fi] (v4) at (2,0) {};
	\node[odd,label=east:$v_5$,fill=\fillve,label=center:\if\showprio1{1}\fi] (v5) at (4,1) {};
	
	\path (v4) edge (v5) ;
	\path (v3) edge [bend right] (v1) ;
	\path (v1) edge [bend right] (v3) ;
	\path (v3) edge (v4) ;
	\path (v2) edge (v4) ;
	\path (v2) edge (v1) ;
	\path (v5) edge (v2) ;
\end{tikzpicture}}\pause%
			\def\token{2}%
			\stepcounter{picite}\only<\thepicite>{\if\token1\def\fillva{tokencolor}\else\def\fillva{bgTUE}\fi%
\if\token2\def\fillvb{tokencolor}\else\def\fillvb{bgTUE}\fi%
\if\token3\def\fillvc{tokencolor}\else\def\fillvc{bgTUE}\fi%
\if\token4\def\fillvd{tokencolor}\else\def\fillvd{bgTUE}\fi%
\if\token5\def\fillve{tokencolor}\else\def\fillve{bgTUE}\fi%
\begin{tikzpicture}[->,scale=0.9, every node/.style={scale=0.8}]
	\tikzstyle{even} = [diamond,draw,minimum size=0.75cm]
	\tikzstyle{odd}  = [rectangle,draw,shape aspect=1,minimum size=0.6cm]
	
	\node[even, label=west:$v_1$,fill=\fillva,label=center:\if\showprio1{3}\fi] (v1) at (0,2) {};
	\node[even, label=east:$v_2$,fill=\fillvb,label=center:\if\showprio1{0}\fi] (v2) at (2,2) {};
	\node[odd,label=west:$v_3$,fill=\fillvc,label=center:\if\showprio1{2}\fi]  (v3) at (0,0) {};
	\node[even,label=east:$v_4$,fill=\fillvd,label=center:\if\showprio1{2}\fi] (v4) at (2,0) {};
	\node[odd,label=east:$v_5$,fill=\fillve,label=center:\if\showprio1{1}\fi] (v5) at (4,1) {};
	
	\path (v4) edge (v5) ;
	\path (v3) edge [bend right] (v1) ;
	\path (v1) edge [bend right] (v3) ;
	\path (v3) edge (v4) ;
	\path (v2) edge (v4) ;
	\path (v2) edge (v1) ;
	\path (v5) edge (v2) ;
\end{tikzpicture}}\pause%
			\def\token{4}%
			\stepcounter{picite}\only<\thepicite>{\if\token1\def\fillva{tokencolor}\else\def\fillva{bgTUE}\fi%
\if\token2\def\fillvb{tokencolor}\else\def\fillvb{bgTUE}\fi%
\if\token3\def\fillvc{tokencolor}\else\def\fillvc{bgTUE}\fi%
\if\token4\def\fillvd{tokencolor}\else\def\fillvd{bgTUE}\fi%
\if\token5\def\fillve{tokencolor}\else\def\fillve{bgTUE}\fi%
\begin{tikzpicture}[->,scale=0.9, every node/.style={scale=0.8}]
	\tikzstyle{even} = [diamond,draw,minimum size=0.75cm]
	\tikzstyle{odd}  = [rectangle,draw,shape aspect=1,minimum size=0.6cm]
	
	\node[even, label=west:$v_1$,fill=\fillva,label=center:\if\showprio1{3}\fi] (v1) at (0,2) {};
	\node[even, label=east:$v_2$,fill=\fillvb,label=center:\if\showprio1{0}\fi] (v2) at (2,2) {};
	\node[odd,label=west:$v_3$,fill=\fillvc,label=center:\if\showprio1{2}\fi]  (v3) at (0,0) {};
	\node[even,label=east:$v_4$,fill=\fillvd,label=center:\if\showprio1{2}\fi] (v4) at (2,0) {};
	\node[odd,label=east:$v_5$,fill=\fillve,label=center:\if\showprio1{1}\fi] (v5) at (4,1) {};
	
	\path (v4) edge (v5) ;
	\path (v3) edge [bend right] (v1) ;
	\path (v1) edge [bend right] (v3) ;
	\path (v3) edge (v4) ;
	\path (v2) edge (v4) ;
	\path (v2) edge (v1) ;
	\path (v5) edge (v2) ;
\end{tikzpicture}}\pause%
			\def\token{5}%
			\stepcounter{picite}\only<\thepicite>{\if\token1\def\fillva{tokencolor}\else\def\fillva{bgTUE}\fi%
\if\token2\def\fillvb{tokencolor}\else\def\fillvb{bgTUE}\fi%
\if\token3\def\fillvc{tokencolor}\else\def\fillvc{bgTUE}\fi%
\if\token4\def\fillvd{tokencolor}\else\def\fillvd{bgTUE}\fi%
\if\token5\def\fillve{tokencolor}\else\def\fillve{bgTUE}\fi%
\begin{tikzpicture}[->,scale=0.9, every node/.style={scale=0.8}]
	\tikzstyle{even} = [diamond,draw,minimum size=0.75cm]
	\tikzstyle{odd}  = [rectangle,draw,shape aspect=1,minimum size=0.6cm]
	
	\node[even, label=west:$v_1$,fill=\fillva,label=center:\if\showprio1{3}\fi] (v1) at (0,2) {};
	\node[even, label=east:$v_2$,fill=\fillvb,label=center:\if\showprio1{0}\fi] (v2) at (2,2) {};
	\node[odd,label=west:$v_3$,fill=\fillvc,label=center:\if\showprio1{2}\fi]  (v3) at (0,0) {};
	\node[even,label=east:$v_4$,fill=\fillvd,label=center:\if\showprio1{2}\fi] (v4) at (2,0) {};
	\node[odd,label=east:$v_5$,fill=\fillve,label=center:\if\showprio1{1}\fi] (v5) at (4,1) {};
	
	\path (v4) edge (v5) ;
	\path (v3) edge [bend right] (v1) ;
	\path (v1) edge [bend right] (v3) ;
	\path (v3) edge (v4) ;
	\path (v2) edge (v4) ;
	\path (v2) edge (v1) ;
	\path (v5) edge (v2) ;
\end{tikzpicture}}\pause%
			\def\token{2}%
			\stepcounter{picite}\only<\thepicite>{\if\token1\def\fillva{tokencolor}\else\def\fillva{bgTUE}\fi%
\if\token2\def\fillvb{tokencolor}\else\def\fillvb{bgTUE}\fi%
\if\token3\def\fillvc{tokencolor}\else\def\fillvc{bgTUE}\fi%
\if\token4\def\fillvd{tokencolor}\else\def\fillvd{bgTUE}\fi%
\if\token5\def\fillve{tokencolor}\else\def\fillve{bgTUE}\fi%
\begin{tikzpicture}[->,scale=0.9, every node/.style={scale=0.8}]
	\tikzstyle{even} = [diamond,draw,minimum size=0.75cm]
	\tikzstyle{odd}  = [rectangle,draw,shape aspect=1,minimum size=0.6cm]
	
	\node[even, label=west:$v_1$,fill=\fillva,label=center:\if\showprio1{3}\fi] (v1) at (0,2) {};
	\node[even, label=east:$v_2$,fill=\fillvb,label=center:\if\showprio1{0}\fi] (v2) at (2,2) {};
	\node[odd,label=west:$v_3$,fill=\fillvc,label=center:\if\showprio1{2}\fi]  (v3) at (0,0) {};
	\node[even,label=east:$v_4$,fill=\fillvd,label=center:\if\showprio1{2}\fi] (v4) at (2,0) {};
	\node[odd,label=east:$v_5$,fill=\fillve,label=center:\if\showprio1{1}\fi] (v5) at (4,1) {};
	
	\path (v4) edge (v5) ;
	\path (v3) edge [bend right] (v1) ;
	\path (v1) edge [bend right] (v3) ;
	\path (v3) edge (v4) ;
	\path (v2) edge (v4) ;
	\path (v2) edge (v1) ;
	\path (v5) edge (v2) ;
\end{tikzpicture}}\pause%
			\def\showprio{1}%
			\def\token{-1}%
			\setcounter{piciteb}{\thepicite}\stepcounter{piciteb}%
			\stepcounter{picite}\only<\thepicite-\thepiciteb>{\if\token1\def\fillva{tokencolor}\else\def\fillva{bgTUE}\fi%
\if\token2\def\fillvb{tokencolor}\else\def\fillvb{bgTUE}\fi%
\if\token3\def\fillvc{tokencolor}\else\def\fillvc{bgTUE}\fi%
\if\token4\def\fillvd{tokencolor}\else\def\fillvd{bgTUE}\fi%
\if\token5\def\fillve{tokencolor}\else\def\fillve{bgTUE}\fi%
\begin{tikzpicture}[->,scale=0.9, every node/.style={scale=0.8}]
	\tikzstyle{even} = [diamond,draw,minimum size=0.75cm]
	\tikzstyle{odd}  = [rectangle,draw,shape aspect=1,minimum size=0.6cm]
	
	\node[even, label=west:$v_1$,fill=\fillva,label=center:\if\showprio1{3}\fi] (v1) at (0,2) {};
	\node[even, label=east:$v_2$,fill=\fillvb,label=center:\if\showprio1{0}\fi] (v2) at (2,2) {};
	\node[odd,label=west:$v_3$,fill=\fillvc,label=center:\if\showprio1{2}\fi]  (v3) at (0,0) {};
	\node[even,label=east:$v_4$,fill=\fillvd,label=center:\if\showprio1{2}\fi] (v4) at (2,0) {};
	\node[odd,label=east:$v_5$,fill=\fillve,label=center:\if\showprio1{1}\fi] (v5) at (4,1) {};
	
	\path (v4) edge (v5) ;
	\path (v3) edge [bend right] (v1) ;
	\path (v1) edge [bend right] (v3) ;
	\path (v3) edge (v4) ;
	\path (v2) edge (v4) ;
	\path (v2) edge (v1) ;
	\path (v5) edge (v2) ;
\end{tikzpicture}}%
			\def\token{1}%
			\stepcounter{picite}\only<\thepicite>{\if\token1\def\fillva{tokencolor}\else\def\fillva{bgTUE}\fi%
\if\token2\def\fillvb{tokencolor}\else\def\fillvb{bgTUE}\fi%
\if\token3\def\fillvc{tokencolor}\else\def\fillvc{bgTUE}\fi%
\if\token4\def\fillvd{tokencolor}\else\def\fillvd{bgTUE}\fi%
\if\token5\def\fillve{tokencolor}\else\def\fillve{bgTUE}\fi%
\begin{tikzpicture}[->,scale=0.9, every node/.style={scale=0.8}]
	\tikzstyle{even} = [diamond,draw,minimum size=0.75cm]
	\tikzstyle{odd}  = [rectangle,draw,shape aspect=1,minimum size=0.6cm]
	
	\node[even, label=west:$v_1$,fill=\fillva,label=center:\if\showprio1{3}\fi] (v1) at (0,2) {};
	\node[even, label=east:$v_2$,fill=\fillvb,label=center:\if\showprio1{0}\fi] (v2) at (2,2) {};
	\node[odd,label=west:$v_3$,fill=\fillvc,label=center:\if\showprio1{2}\fi]  (v3) at (0,0) {};
	\node[even,label=east:$v_4$,fill=\fillvd,label=center:\if\showprio1{2}\fi] (v4) at (2,0) {};
	\node[odd,label=east:$v_5$,fill=\fillve,label=center:\if\showprio1{1}\fi] (v5) at (4,1) {};
	
	\path (v4) edge (v5) ;
	\path (v3) edge [bend right] (v1) ;
	\path (v1) edge [bend right] (v3) ;
	\path (v3) edge (v4) ;
	\path (v2) edge (v4) ;
	\path (v2) edge (v1) ;
	\path (v5) edge (v2) ;
\end{tikzpicture}}%
			\def\token{3}%
			\stepcounter{picite}\only<\thepicite>{\if\token1\def\fillva{tokencolor}\else\def\fillva{bgTUE}\fi%
\if\token2\def\fillvb{tokencolor}\else\def\fillvb{bgTUE}\fi%
\if\token3\def\fillvc{tokencolor}\else\def\fillvc{bgTUE}\fi%
\if\token4\def\fillvd{tokencolor}\else\def\fillvd{bgTUE}\fi%
\if\token5\def\fillve{tokencolor}\else\def\fillve{bgTUE}\fi%
\begin{tikzpicture}[->,scale=0.9, every node/.style={scale=0.8}]
	\tikzstyle{even} = [diamond,draw,minimum size=0.75cm]
	\tikzstyle{odd}  = [rectangle,draw,shape aspect=1,minimum size=0.6cm]
	
	\node[even, label=west:$v_1$,fill=\fillva,label=center:\if\showprio1{3}\fi] (v1) at (0,2) {};
	\node[even, label=east:$v_2$,fill=\fillvb,label=center:\if\showprio1{0}\fi] (v2) at (2,2) {};
	\node[odd,label=west:$v_3$,fill=\fillvc,label=center:\if\showprio1{2}\fi]  (v3) at (0,0) {};
	\node[even,label=east:$v_4$,fill=\fillvd,label=center:\if\showprio1{2}\fi] (v4) at (2,0) {};
	\node[odd,label=east:$v_5$,fill=\fillve,label=center:\if\showprio1{1}\fi] (v5) at (4,1) {};
	
	\path (v4) edge (v5) ;
	\path (v3) edge [bend right] (v1) ;
	\path (v1) edge [bend right] (v3) ;
	\path (v3) edge (v4) ;
	\path (v2) edge (v4) ;
	\path (v2) edge (v1) ;
	\path (v5) edge (v2) ;
\end{tikzpicture}}%
			\def\token{1}%
			\stepcounter{picite}\only<\thepicite>{\if\token1\def\fillva{tokencolor}\else\def\fillva{bgTUE}\fi%
\if\token2\def\fillvb{tokencolor}\else\def\fillvb{bgTUE}\fi%
\if\token3\def\fillvc{tokencolor}\else\def\fillvc{bgTUE}\fi%
\if\token4\def\fillvd{tokencolor}\else\def\fillvd{bgTUE}\fi%
\if\token5\def\fillve{tokencolor}\else\def\fillve{bgTUE}\fi%
\begin{tikzpicture}[->,scale=0.9, every node/.style={scale=0.8}]
	\tikzstyle{even} = [diamond,draw,minimum size=0.75cm]
	\tikzstyle{odd}  = [rectangle,draw,shape aspect=1,minimum size=0.6cm]
	
	\node[even, label=west:$v_1$,fill=\fillva,label=center:\if\showprio1{3}\fi] (v1) at (0,2) {};
	\node[even, label=east:$v_2$,fill=\fillvb,label=center:\if\showprio1{0}\fi] (v2) at (2,2) {};
	\node[odd,label=west:$v_3$,fill=\fillvc,label=center:\if\showprio1{2}\fi]  (v3) at (0,0) {};
	\node[even,label=east:$v_4$,fill=\fillvd,label=center:\if\showprio1{2}\fi] (v4) at (2,0) {};
	\node[odd,label=east:$v_5$,fill=\fillve,label=center:\if\showprio1{1}\fi] (v5) at (4,1) {};
	
	\path (v4) edge (v5) ;
	\path (v3) edge [bend right] (v1) ;
	\path (v1) edge [bend right] (v3) ;
	\path (v3) edge (v4) ;
	\path (v2) edge (v4) ;
	\path (v2) edge (v1) ;
	\path (v5) edge (v2) ;
\end{tikzpicture}}%
			\def\token{-1}%
			\stepcounter{picite}\only<\thepicite-|handout:0>{\if\token1\def\fillva{tokencolor}\else\def\fillva{bgTUE}\fi%
\if\token2\def\fillvb{tokencolor}\else\def\fillvb{bgTUE}\fi%
\if\token3\def\fillvc{tokencolor}\else\def\fillvc{bgTUE}\fi%
\if\token4\def\fillvd{tokencolor}\else\def\fillvd{bgTUE}\fi%
\if\token5\def\fillve{tokencolor}\else\def\fillve{bgTUE}\fi%
\begin{tikzpicture}[->,scale=0.9, every node/.style={scale=0.8}]
	\tikzstyle{even} = [diamond,draw,minimum size=0.75cm]
	\tikzstyle{odd}  = [rectangle,draw,shape aspect=1,minimum size=0.6cm]
	
	\node[even, label=west:$v_1$,fill=\fillva,label=center:\if\showprio1{3}\fi] (v1) at (0,2) {};
	\node[even, label=east:$v_2$,fill=\fillvb,label=center:\if\showprio1{0}\fi] (v2) at (2,2) {};
	\node[odd,label=west:$v_3$,fill=\fillvc,label=center:\if\showprio1{2}\fi]  (v3) at (0,0) {};
	\node[even,label=east:$v_4$,fill=\fillvd,label=center:\if\showprio1{2}\fi] (v4) at (2,0) {};
	\node[odd,label=east:$v_5$,fill=\fillve,label=center:\if\showprio1{1}\fi] (v5) at (4,1) {};
	
	\path (v4) edge (v5) ;
	\path (v3) edge [bend right] (v1) ;
	\path (v1) edge [bend right] (v3) ;
	\path (v3) edge (v4) ;
	\path (v2) edge (v4) ;
	\path (v2) edge (v1) ;
	\path (v5) edge (v2) ;
\end{tikzpicture}}%
		\end{center}
		\begin{itemize}
			\item The winner is determined by the parity of the priority occurring infinitely often\pause
			\pause\pause\pause%display game
			\item Player 1 wins $\{v_1,v_3\}$, using $v_3 \mapsto v_1$\pause
			\item Player 0 wins $\{v_2,v_4,v_5\}$, using $v_2 \mapsto v_4$\pause\stepcounter{picite}
			\item \textit{Solving}: Partition the vertices in $W_0,W_1$\pause\stepcounter{picite}
		\end{itemize}
	\end{column}
	\begin{column}{0.5\textwidth}
		\large Variability parity game.\small\\
		Configurations $\mathfrak{C} = \{c_1,c_2\}$\\
		\begin{center}
			\def\showprio{1}%
			\def\token{-1}%%%
			\def\playconf{0}%
			\stepcounter{picite}\only<\thepicite>{\input{tikz_vpg_example}}%
			\def\playconf{1}%
			\stepcounter{picite}\only<\thepicite>{\input{tikz_vpg_example}}%
			\def\playconf{2}%
			\stepcounter{picite}\only<\thepicite>{\input{tikz_vpg_example}}%
			\def\playconf{0}%
			\stepcounter{picite}\only<\thepicite>{\input{tikz_vpg_example}}%
		\end{center}
		\begin{itemize}\pause
			\item $W_0^{c_1} = \emptyset$,$W_1^{c_1} = \{v_1,\dots,v_5\}$\pause
			\item $W_0^{c_2} = \{v_1,v_3\}$,$W_1^{c_2} = \{v_2,v_4,v_5\}$\pause
			\item \textit{Solving}: Partition the vertices in $W_0^c,W_1^c$, for every $c \in \mathfrak{C}$
		\end{itemize}
	\end{column}
\end{columns}
\end{frame}
%----------------------------------------------------------------------------------------

\begin{frame}[t]
\frametitle{Variability parity game}
\begin{columns}[t]
	\begin{column}{0.5\textwidth}
		\begin{center}
			\setcounter{picite}{0}%
			\stepcounter{picite}\only<\thepicite>{\begin{tikzpicture}[->]
			\tikzstyle{even} = [diamond,draw,minimum size=0.75cm]
			\tikzstyle{odd}  = [rectangle,draw,shape aspect=1,minimum size=0.75cm]
			
			\node (lts) at (1,3.5)  {LTS $M$};
			\node (form)at (5,3.5)  {Modal $\mu$-calculus formula $\varphi$};
			\node (check)at (3,1.5){$M$ satisfies $\varphi$?};
			
			\path[-] (lts) edge (3,2.5);
			\path[-] (form)edge (3,2.5);
			\path (3,2.5)  edge (check);
			\end{tikzpicture}}\pause
			\setcounter{piciteb}{\thepicite}%
			\stepcounter{piciteb}\stepcounter{piciteb}\stepcounter{piciteb}%
			\stepcounter{picite}\only<\thepicite-\thepiciteb>{\begin{tikzpicture}[->]
				\tikzstyle{even} = [diamond,draw,minimum size=0.75cm]
				\tikzstyle{odd}  = [rectangle,draw,shape aspect=1,minimum size=0.75cm]
				
				\node (lts) at (1,3.5)  {LTS $M$};
				\node (form)at (5,3.5)  {Modal $\mu$-calculus formula $\varphi$};
				\node (PG)at (3,2.5) {PG};
				\node (check)at (3,1.5){$M$ satisfies $\varphi$?};
				
				\path[-] (lts) edge (PG);
				\path[-] (form)edge (PG);
				\path (PG)  edge (check);
			\end{tikzpicture}}
			\begin{theorem}
				A parity game can be constructed from an LTS and a modal $\mu$-calculus formula $\varphi$ such that $M$ satisfies $\varphi$ iff special vertex $v_0 \in W_0$ in the resulting parity game.
			\end{theorem}
		\end{center}
	\end{column}
	\begin{column}{0.5\textwidth}\pause
		\begin{center}
			\stepcounter{picite}\only<\thepicite>{\begin{tikzpicture}[->]
				\tikzstyle{even} = [diamond,draw,minimum size=0.75cm]
				\tikzstyle{odd}  = [rectangle,draw,shape aspect=1,minimum size=0.75cm]
				
				\node (lts) at (1,3.5)  {FTS $M$};
				\node (form)at (5,3.5)  {Modal $\mu$-calculus formula $\varphi$};
				\node (check)at (3,1.5){Which products in $M$ satisfy $\varphi$?};
				
				\path[-] (lts) edge (3,2.5);
				\path[-] (form)edge (3,2.5);
				\path (3,2.5)  edge (check);
			\end{tikzpicture}}\pause
			\setcounter{piciteb}{\thepicite}%
			\stepcounter{piciteb}%
			\stepcounter{picite}\only<\thepicite>{\begin{tikzpicture}[->]
			\tikzstyle{even} = [diamond,draw,minimum size=0.75cm]
			\tikzstyle{odd}  = [rectangle,draw,shape aspect=1,minimum size=0.75cm]
			
			\node (lts) at (1,3.5)  {FTS $M$};
			\node (form)at (5,3.5)  {Modal $\mu$-calculus formula $\varphi$};
			\node (VPG)at (3,2.5) {VPG};
			\node (check)at (3,1.5){Which products in $M$ satisfy $\varphi$?};
			
			\path[-] (lts) edge (VPG);
			\path[-] (form)edge (VPG);
			\path (VPG)  edge (check);
			\end{tikzpicture}}
			\begin{theorem}
				A VPG can be constructed from an FTS and a modal $\mu$-calculus formula $\varphi$ such that $M$ satisfies $\varphi$ for product $p$ iff special vertex $v_0 \in W_0^p$ in the resulting VPG.
			\end{theorem}\pause
			\stepcounter{piciteb}\only<\thepiciteb>{\begin{tikzpicture}
				\node (FTSM) at (10,10) {FTS};
				\node[opacity=0] (VPGG) at (15,10) {VPG};
				\node (LTSMp) at (10,8) {LTS};
				\node (PGGp) at (15,8) {PG};
				
				\path[->,opacity=0] (FTSM) edge node[above]{$\varphi$} (VPGG);
				\path[->] (LTSMp) edge node[above]{$\varphi$} (PGGp);
				\path[->,dashed] (FTSM) edge node[left]{$\Pi_p$} (LTSMp);
				\path[->,dashed,opacity=0] (VPGG) edge node[right]{$\Pi_p$} (PGGp);
			\end{tikzpicture}}
			\stepcounter{piciteb}\only<\thepiciteb>{\begin{tikzpicture}
				\node (FTSM) at (10,10) {FTS};
				\node (VPGG) at (15,10) {VPG};
				\node (LTSMp) at (10,8) {LTS};
				\node (PGGp) at (15,8) {PG};
				
				\path[->] (FTSM) edge node[above]{$\varphi$} (VPGG);
				\path[->] (LTSMp) edge node[above]{$\varphi$} (PGGp);
				\path[->,dashed] (FTSM) edge node[left]{$\Pi_p$} (LTSMp);
				\path[->,dashed] (VPGG) edge node[right]{$\Pi_p$} (PGGp);
			\end{tikzpicture}}
		\end{center}
	\end{column}
\end{columns}
\end{frame}

%----------------------------------------------------------------------------------------
\begin{frame}[t]
\frametitle{VPG algorithms - Recursive algorithm}
\begin{columns}[t]
	\begin{column}{0.5\textwidth}
		\begin{itemize}
			\item The recursive algorithm reasons about sets of vertices
		\end{itemize}
		\def\highlighteda{0}%
		\def\highlightedb{0}%
		\def\highlightedc{0}%
		\def\highlightedd{1}%
		\def\highlightede{0}%
		\def\highlightedf{0}%
		\def\highlightedg{0}%
		\def\highlightedh{1}%
		
		\setcounter{picite}{0}%
		\stepcounter{picite}\only<\thepicite>{\if\highlighteda1\def\fillva{highlightgraph}\else\def\fillva{bgTUE}\fi%
\if\highlightedb1\def\fillvb{highlightgraph}\else\def\fillvb{bgTUE}\fi%
\if\highlightedc1\def\fillvc{highlightgraph}\else\def\fillvc{bgTUE}\fi%
\if\highlightedd1\def\fillvd{highlightgraph}\else\def\fillvd{bgTUE}\fi%
\if\highlightede1\def\fillve{highlightgraph}\else\def\fillve{bgTUE}\fi%
\if\highlightedf1\def\fillvf{highlightgraph}\else\def\fillvf{bgTUE}\fi%
\if\highlightedg1\def\fillvg{highlightgraph}\else\def\fillvg{bgTUE}\fi%
\if\highlightedh1\def\fillvh{highlightgraph}\else\def\fillvh{bgTUE}\fi%
\begin{tikzpicture}[->,scale=0.9, every node/.style={scale=0.8}]
	\tikzstyle{even} = [diamond,draw,minimum size=0.75cm]
	\tikzstyle{odd}  = [rectangle,draw,shape aspect=1,minimum size=0.6cm]
	
	\node[even,label=center:1,fill=\fillva] (v1) at (10,10) {};
	\node[odd,label=center:3,fill=\fillvb]  (v2) at (11.5,10) {};
	\node[odd,label=center:3,fill=\fillvc]  (v3) at (13,10) {};
	\node[even,label=center:2,fill=\fillvd] (v4) at (14.5,10) {};
	\node[even,label=center:2,fill=\fillve] (v5) at (16,10) {};
	
	\node[even,label=center:1,fill=\fillvf] (v6) at (11.5,8.5) {};
	\node[odd,label=center:0,fill=\fillvg]  (v7) at (13,8.5) {};
	\node[even,label=center:0,fill=\fillvh] (v8) at (14.5,8.5) {};
	
	\path (v1) edge [bend right] (v2);
	\path (v2) edge [bend right] (v1);
	\path (v2) edge (v3);
	\path (v3) edge (v4);
	\path (v4) edge [bend left] (v5);
	\path (v5) edge [bend left] (v4);
	
	\path (v6) edge (v1);
	\path (v2) edge (v6);
	\path (v6) edge (v3);
	\path (v7) edge (v3);
	\path (v7) edge (v8);
	\path (v7) edge (v6);
	\path (v3) edge (v8);
	\path (v4) edge [bend left] (v8);
	\path (v8) edge [bend left] (v4);
\end{tikzpicture}}\pause%
		\def\highlightede{1}%
		\def\highlightedc{1}%
		\stepcounter{picite}\only<\thepicite>{\if\highlighteda1\def\fillva{highlightgraph}\else\def\fillva{bgTUE}\fi%
\if\highlightedb1\def\fillvb{highlightgraph}\else\def\fillvb{bgTUE}\fi%
\if\highlightedc1\def\fillvc{highlightgraph}\else\def\fillvc{bgTUE}\fi%
\if\highlightedd1\def\fillvd{highlightgraph}\else\def\fillvd{bgTUE}\fi%
\if\highlightede1\def\fillve{highlightgraph}\else\def\fillve{bgTUE}\fi%
\if\highlightedf1\def\fillvf{highlightgraph}\else\def\fillvf{bgTUE}\fi%
\if\highlightedg1\def\fillvg{highlightgraph}\else\def\fillvg{bgTUE}\fi%
\if\highlightedh1\def\fillvh{highlightgraph}\else\def\fillvh{bgTUE}\fi%
\begin{tikzpicture}[->,scale=0.9, every node/.style={scale=0.8}]
	\tikzstyle{even} = [diamond,draw,minimum size=0.75cm]
	\tikzstyle{odd}  = [rectangle,draw,shape aspect=1,minimum size=0.6cm]
	
	\node[even,label=center:1,fill=\fillva] (v1) at (10,10) {};
	\node[odd,label=center:3,fill=\fillvb]  (v2) at (11.5,10) {};
	\node[odd,label=center:3,fill=\fillvc]  (v3) at (13,10) {};
	\node[even,label=center:2,fill=\fillvd] (v4) at (14.5,10) {};
	\node[even,label=center:2,fill=\fillve] (v5) at (16,10) {};
	
	\node[even,label=center:1,fill=\fillvf] (v6) at (11.5,8.5) {};
	\node[odd,label=center:0,fill=\fillvg]  (v7) at (13,8.5) {};
	\node[even,label=center:0,fill=\fillvh] (v8) at (14.5,8.5) {};
	
	\path (v1) edge [bend right] (v2);
	\path (v2) edge [bend right] (v1);
	\path (v2) edge (v3);
	\path (v3) edge (v4);
	\path (v4) edge [bend left] (v5);
	\path (v5) edge [bend left] (v4);
	
	\path (v6) edge (v1);
	\path (v2) edge (v6);
	\path (v6) edge (v3);
	\path (v7) edge (v3);
	\path (v7) edge (v8);
	\path (v7) edge (v6);
	\path (v3) edge (v8);
	\path (v4) edge [bend left] (v8);
	\path (v8) edge [bend left] (v4);
\end{tikzpicture}}\pause%
		\def\highlightedf{1}%
		\stepcounter{picite}\only<\thepicite>{\if\highlighteda1\def\fillva{highlightgraph}\else\def\fillva{bgTUE}\fi%
\if\highlightedb1\def\fillvb{highlightgraph}\else\def\fillvb{bgTUE}\fi%
\if\highlightedc1\def\fillvc{highlightgraph}\else\def\fillvc{bgTUE}\fi%
\if\highlightedd1\def\fillvd{highlightgraph}\else\def\fillvd{bgTUE}\fi%
\if\highlightede1\def\fillve{highlightgraph}\else\def\fillve{bgTUE}\fi%
\if\highlightedf1\def\fillvf{highlightgraph}\else\def\fillvf{bgTUE}\fi%
\if\highlightedg1\def\fillvg{highlightgraph}\else\def\fillvg{bgTUE}\fi%
\if\highlightedh1\def\fillvh{highlightgraph}\else\def\fillvh{bgTUE}\fi%
\begin{tikzpicture}[->,scale=0.9, every node/.style={scale=0.8}]
	\tikzstyle{even} = [diamond,draw,minimum size=0.75cm]
	\tikzstyle{odd}  = [rectangle,draw,shape aspect=1,minimum size=0.6cm]
	
	\node[even,label=center:1,fill=\fillva] (v1) at (10,10) {};
	\node[odd,label=center:3,fill=\fillvb]  (v2) at (11.5,10) {};
	\node[odd,label=center:3,fill=\fillvc]  (v3) at (13,10) {};
	\node[even,label=center:2,fill=\fillvd] (v4) at (14.5,10) {};
	\node[even,label=center:2,fill=\fillve] (v5) at (16,10) {};
	
	\node[even,label=center:1,fill=\fillvf] (v6) at (11.5,8.5) {};
	\node[odd,label=center:0,fill=\fillvg]  (v7) at (13,8.5) {};
	\node[even,label=center:0,fill=\fillvh] (v8) at (14.5,8.5) {};
	
	\path (v1) edge [bend right] (v2);
	\path (v2) edge [bend right] (v1);
	\path (v2) edge (v3);
	\path (v3) edge (v4);
	\path (v4) edge [bend left] (v5);
	\path (v5) edge [bend left] (v4);
	
	\path (v6) edge (v1);
	\path (v2) edge (v6);
	\path (v6) edge (v3);
	\path (v7) edge (v3);
	\path (v7) edge (v8);
	\path (v7) edge (v6);
	\path (v3) edge (v8);
	\path (v4) edge [bend left] (v8);
	\path (v8) edge [bend left] (v4);
\end{tikzpicture}}\pause%
		\def\highlightedg{1}%
		\stepcounter{picite}\only<\thepicite->{\if\highlighteda1\def\fillva{highlightgraph}\else\def\fillva{bgTUE}\fi%
\if\highlightedb1\def\fillvb{highlightgraph}\else\def\fillvb{bgTUE}\fi%
\if\highlightedc1\def\fillvc{highlightgraph}\else\def\fillvc{bgTUE}\fi%
\if\highlightedd1\def\fillvd{highlightgraph}\else\def\fillvd{bgTUE}\fi%
\if\highlightede1\def\fillve{highlightgraph}\else\def\fillve{bgTUE}\fi%
\if\highlightedf1\def\fillvf{highlightgraph}\else\def\fillvf{bgTUE}\fi%
\if\highlightedg1\def\fillvg{highlightgraph}\else\def\fillvg{bgTUE}\fi%
\if\highlightedh1\def\fillvh{highlightgraph}\else\def\fillvh{bgTUE}\fi%
\begin{tikzpicture}[->,scale=0.9, every node/.style={scale=0.8}]
	\tikzstyle{even} = [diamond,draw,minimum size=0.75cm]
	\tikzstyle{odd}  = [rectangle,draw,shape aspect=1,minimum size=0.6cm]
	
	\node[even,label=center:1,fill=\fillva] (v1) at (10,10) {};
	\node[odd,label=center:3,fill=\fillvb]  (v2) at (11.5,10) {};
	\node[odd,label=center:3,fill=\fillvc]  (v3) at (13,10) {};
	\node[even,label=center:2,fill=\fillvd] (v4) at (14.5,10) {};
	\node[even,label=center:2,fill=\fillve] (v5) at (16,10) {};
	
	\node[even,label=center:1,fill=\fillvf] (v6) at (11.5,8.5) {};
	\node[odd,label=center:0,fill=\fillvg]  (v7) at (13,8.5) {};
	\node[even,label=center:0,fill=\fillvh] (v8) at (14.5,8.5) {};
	
	\path (v1) edge [bend right] (v2);
	\path (v2) edge [bend right] (v1);
	\path (v2) edge (v3);
	\path (v3) edge (v4);
	\path (v4) edge [bend left] (v5);
	\path (v5) edge [bend left] (v4);
	
	\path (v6) edge (v1);
	\path (v2) edge (v6);
	\path (v6) edge (v3);
	\path (v7) edge (v3);
	\path (v7) edge (v8);
	\path (v7) edge (v6);
	\path (v3) edge (v8);
	\path (v4) edge [bend left] (v8);
	\path (v8) edge [bend left] (v4);
\end{tikzpicture}}\pause%
	\end{column}
	\begin{column}{0.5\textwidth}
		\begin{itemize}
			\item the recursive algorithm for VPGs reasons about sets of vertex configuration pairs
			\item Attractor calculation example on VPG
			\item Function-wise representation to efficiently perform attractor calcs
			\item Short explanation of symbolic representation
			\item Time complexities
		\end{itemize}
	\end{column}
\end{columns}
\end{frame}

%----------------------------------------------------------------------------------------

\begin{frame}[t]
\frametitle{VPG algorithms - Incremental pre-solve algorithm}
\begin{columns}[t]
	\begin{column}{0.5\textwidth}
		\begin{itemize}
			\item Introduce algorithm;, idea of pre-solving
			\item Introduce pessimistic PGs
			\item We need an alg to solve PGs using pre-solved vertices for efficiency
		\end{itemize}
	\end{column}
	\begin{column}{0.5\textwidth}
	\end{column}
\end{columns}
\end{frame}

%----------------------------------------------------------------------------------------

\begin{frame}[t]
\frametitle{VPG algorithms - Incremental pre-solve algorithm}
\begin{columns}[t]
	\begin{column}{0.5\textwidth}
		\begin{itemize}
			\item FPIte, show FP formula
			\item Show modified FP formula
			\item Explain the efficiency gained
		\end{itemize}
	\end{column}
	\begin{column}{0.5\textwidth}
		\begin{itemize}
			\item Very short explanation of a fixed-point
		\end{itemize}
	\end{column}
\end{columns}
\end{frame}

%----------------------------------------------------------------------------------------

\begin{frame}[t]
\frametitle{VPG algorithms - Local solving}
\begin{columns}[t]
	\begin{column}{0.5\textwidth}
		\begin{itemize}
			\item explain local solving
			\item introduced local algs for the novel VPG algs and existing PG algs.
		\end{itemize}
	\end{column}
	\begin{column}{0.5\textwidth}
	\end{column}
\end{columns}
\end{frame}

%----------------------------------------------------------------------------------------

\begin{frame}[t]
\frametitle{Experimental results - SPL games}
\def\scalegraphs{0.6}
\begin{columns}[t]
	\begin{column}{0.5\textwidth}
		\begin{figure}[H]
			\input{"../results/minepump/Zlnk product based_Zlnk fam based - explicit_Zlnk fam based - symbolic_"}
			\caption{Running times, in ms, on the minepump games.}
			\label{fig:results_minepump}
		\end{figure}%
	\end{column}
	\begin{column}{0.5\textwidth}
		\begin{figure}[H]
			\input{"../results/elevator/Zlnk product based_Zlnk fam based - explicit_Zlnk fam based - symbolic_"}
			\caption{Running times, in ms, on the elevator games.}
			\label{fig:results_elevator}
		\end{figure}%
	\end{column}
\end{columns}
\small
\raisebox{.7\height}{\begin{tikzpicture}
\path[line width=2pt,color=cyan] (19,20) edge (20,20);
\end{tikzpicture}} Recursive algorithm for parity games\\
\raisebox{.7\height}{\begin{tikzpicture}
\path[line width=2pt,color=green] (19,20) edge (20,20);
\end{tikzpicture}} Recursive algorithm for VPGs with a symbolic representation of configurations\\
\raisebox{.7\height}{\begin{tikzpicture}
\path[line width=2pt,color=red] (19,20) edge (20,20);
\end{tikzpicture}} Recursive algorithm for VPGs with an explicit representation of configurations\\
\end{frame}

%----------------------------------------------------------------------------------------

\begin{frame}[t]
\frametitle{Experimental results - SPL games}
\def\scalegraphs{0.6}
\begin{columns}[t]
	\begin{column}{0.5\textwidth}
		\begin{figure}[H]
			\input{"../results/minepump/Fixed-point product based_Incremental pre-solve_"}
			\caption{Running times, in ms, on the minepump games.}
			\label{fig:results_minepump}
		\end{figure}%
	\end{column}
	\begin{column}{0.5\textwidth}
		\begin{figure}[H]
			\input{"../results/elevator/Fixed-point product based_Incremental pre-solve_"}
			\caption{Running times, in ms, on the elevator games.}
			\label{fig:results_elevator}
		\end{figure}%
	\end{column}
\end{columns}
\small
\raisebox{.7\height}{\begin{tikzpicture}
	\path[line width=2pt,color=violet] (19,20) edge (20,20);
	\end{tikzpicture}} Fixed-point iteration algorithm for parity games\\
\raisebox{.7\height}{\begin{tikzpicture}
	\path[line width=2pt,color=orange] (19,20) edge (20,20);
	\end{tikzpicture}} Incremental pre-solve algorithm\\
\end{frame}

%----------------------------------------------------------------------------------------

\begin{frame}[t]
\frametitle{Experimental results - Random games}
\begin{columns}[t]
	\begin{column}{0.5\textwidth}
		\begin{itemize}
			\item Show the type of games where recursive symbolic fails and the explicit does not.
		\end{itemize}
	\end{column}
	\begin{column}{0.5\textwidth}
	\end{column}
\end{columns}
\end{frame}

%----------------------------------------------------------------------------------------

\begin{frame}[t]
\frametitle{Experimental results - Local solving}
\begin{columns}[t]
	\begin{column}{0.5\textwidth}
		\begin{itemize}
			\item Show the same graphs but with local solving as well
		\end{itemize}
	\end{column}
	\begin{column}{0.5\textwidth}
	\end{column}
\end{columns}
\end{frame}

%----------------------------------------------------------------------------------------

\begin{frame}[t]
\frametitle{Conclusions}
\begin{columns}[t]
	\begin{column}{0.5\textwidth}
		\begin{itemize}
			\item Collective approach can improve SPLs verifying performance
			\item The symbolic recursive can do this well
			\item The explicit recursive is "robust"
			\item Local solving can increase performance, however very dependent on alg \& type of VPG
		\end{itemize}
	\end{column}
	\begin{column}{0.5\textwidth}
	\end{column}
\end{columns}
\end{frame}

%----------------------------------------------------------------------------------------

\end{document}