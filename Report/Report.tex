\documentclass[]{article}
\usepackage[margin=1in]{geometry}
\usepackage{amsthm}
\usepackage{mathtools}
\usepackage{amsfonts}


\theoremstyle{definition}
\newtheorem{definition}{Definition}[section]
\newtheorem{theorem}{Theorem}[section]
\newtheorem{lemma}[theorem]{Lemma}



%opening
\title{Verifying Featured Transition Systems using Variability Parity Games}
\author{Sjef van Loo}

\begin{document}

\maketitle

\section{Definitions}
\subsection{Transition systems}
From \cite{Classen2013FeaturedTS}.

\begin{definition}An LTS is a tuple $M = (S, Act, trans, I, AP, L)$, where:
	\begin{itemize}
		\item $S$ is a set of states,
		\item $Act$ a set of actions,
		\item $trans \subseteq S \times Act \times S$ is the transition relation with $(s,a,s') \in trans$ denoted as $s \xrightarrow a s'$,
		\item $I \subseteq S$ is a set of initial states,
		\item $AP$ is a set of atomic propositions, and
		\item $L : S \rightarrow 2^{AP}$ is a labelling function.
	\end{itemize}
\end{definition}

\begin{definition}An FTS is a tuple $M = (S, Act, trans, I, AP, L, N, px, \gamma)$, where:
	\begin{itemize}
		\item $S, Act, trans, I, AP, L$ are defined as in an LTS,
		\item $N$ is a set of features,
		\item $px \subseteq \mathcal{P}(N)$ is a set of products, ie. feature assignments, that are valid,
		\item $\gamma : trans \rightarrow \mathbb{B}(N)$ is a total function, labelling each transition with a Boolean expression over the features. A product $p \in \mathcal{P}(N)$ satisfying the Boolean expression of transition $t$ is denoted as $p \models \gamma(t)$, $\gamma(t)(p) = 1$ or $p \in [\![\gamma(t)]\!]$. 
		
		A transition $s \xrightarrow a s'$ and $\gamma((s,a,s')) = f$ is denoted as $s \xrightarrow {a \backslash f} s'$. 
	\end{itemize}
\end{definition}

\begin{definition}
	The projection of an FTS fts to a product $p \in px$, noted $fts_{|p}$, is the LTS $t=(S,Act,trans', I, AP, L)$, where $trans' = \{t \in trans\ |\ p \models \gamma(t)\}$.
\end{definition}

\section{Variability Parity Games}
\subsection{Option 1}
\begin{definition}
	A Variability Parity Game is a tuple $VG = (V,V_0, V_1, E, \rho, N, \gamma)$, where:
	\begin{itemize}
		\item $V = V_0 \cup V_1$,
		\item $V_0 \cap V_1 = \emptyset$,
		\item $V_0$ is the set of vertices for player $0$,
		\item $V_1$ is the set of vertices for player $1$, 
		\item $E \subseteq V \times V$ is the edge relation,
		\item $\rho :  V \rightarrow \mathbb{N}$ is a priority assignment,
		\item $N$ is a set of features,
		\item $\gamma : E \rightarrow \mathbb{B}(N)$ is a total function, labelling each edge with a Boolean expression over the features.
	\end{itemize}
\end{definition}
A VPG is played for a specific $p \subseteq N$. A path $\pi$ is valid if for all pairs $\pi_i$ and  $\pi_{i+1})$ we have $(\pi_i, \pi_{i+1}) \in E$ and $p \models \gamma((\pi_i, \pi_{i+1}))$.

Not deadlock free, so player $\alpha \in \{0,1\}$ wins iff $\overline{\alpha}$ can't make a move or if the highest priority occurring infinitely often has the same parity as the player.

For a $p \subseteq N$ we have winning sets $W_0^p$ and $W_1^p$.

\begin{definition}
	The projection of a VPG vpg to a product $p \subseteq N$, noted $vpg_{|p}$, is the PG $pg = (V, V_0, V_1, E', \rho)$, where $E' = \{ e \in E\ |\ p \models \gamma(e)\}$.
\end{definition}

\begin{definition}
	FTS2VPG(fts, $\varphi$) converts an FTS and a formula to a VPG. 
	
	...
\end{definition}

\begin{definition}
LTS2PG(lts, $\varphi$) converts an LTS and a formula to a PG.

Use some existing definition.
\end{definition}

\begin{theorem}
	\label{vpgproj}
	$W_\alpha^p$ for VPG vpg is equal to $W_\alpha$ in $vpg_{|p}$ for any $p \subseteq N$ and $\alpha \in \{0,1\}$.
\end{theorem}

From this it theorem follows that the VPG is positionally determined and that the winning sets cover the entire graph.

\begin{lemma}
	\label{vpgftsproj}
	FTS2VPG$(fts, \varphi)_{|p}$ is equal to LTS2PG$(fts_{|p},\varphi)$ for any $p \subseteq N$.
\end{lemma}

\begin{theorem}
	Given FTS $fts = (S, Act, trans, I, AP, L, N, px, \gamma)$ and formula $\varphi$.\\
	For any product $p \in px$ and state $s \in S$ we have:\\
	$fts$ satisfies $\varphi$ for product $p$ in state $s$ iff $s \in W_0^p$ in FTS2VPG(fts, $\varphi$).
\begin{proof}
	
	Winning set $W_0^p$ in FTS2VPG($fts, \varphi$) is equal to winning set $W_0$ in FTS2VPG($fts, \varphi$)$_{|p}$ (using theorem \ref{vpgproj}). Furthermore FTS2VPG($fts, \varphi$)$_{|p}$ is equal to LTS2PG$(fts_{|p}, \varphi)$ (using lemma \ref{vpgftsproj}).
	
	So winning set  $W_0^p$ in FTS2VPG($fts, \varphi$) is equal to winning set $W_0$ in LTS2PG$(fts_{|p}, \varphi)$.
	Since $fts_{|p}$ satisfies $\varphi$ in state $s$ iff $s \in W_0$ in LTS2PG$(fts_{|p}, \varphi)$ (existing LTS verification theory) the theorem holds.
\end{proof}
\end{theorem}
\bibliography{mybib} 
\bibliographystyle{ieeetr}

\end{document}
